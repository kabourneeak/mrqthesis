\chapter{Conclusion}
\label{:conclusion}

In this final chapter, we summarize the contributions from the previous chapters as well as possible directions for future research.

\section{Summary of Contributions}
\label{:conclusion:contributions}

This thesis presents several contributions in the area of partial enclosure range searching.

In Chapter~\ref{:rectangles}, we developed methods for querying line segments with axis-parallel rectangles. 
We first considered axis-parallel segments and gave a method for expressing the partial enclosure property as an orthogonal range query. 
Next we considered arbitrarily-oriented line segments and we saw that the uncertainty of where a segment may intersect the query significantly increases the complexity of testing partial enclosure expressions.

In Chapter~\ref{:slabs}, we consider axis-parallel line segments against a query region which is either an arbitrarily-oriented slab, or the intersection of two such slabs; the latter is a generalization of querying with an arbitrarily-oriented rectangle. 
We gave a method for expressing the partial enclosure property as a half-plane query in a dual-space.

In Chapter~\ref{:convexp}, we moved away from line segments to consider convex polygons and partial enclosure range searches involving area. 
We developed a method for preprocessing a convex polygon such that the area to one side of any chord can be found in logarithmic time with respect to the number of edges in the polygon. 
We gave an algorithm for decomposing the intersection of a rectangle with a convex polygon in such a way that the area of their intersection could be found in logarithmic time. 

In Chapter~\ref{:monotonep}, we considered monotone polygons. 
We gave an algorithm which produces a list of area formulas that can be queried for the area of the polygon underneath a horizontal query line.
We combined this approach with a recursive decomposition of the polygon to create a method for calculating the area of a monotone polygon within a query rectangle in logarithmic time with respect to the number of vertices in the polygon. 


\section{Future Work}
\label{:conclusion:open-problems}

In this section, we summarize the future work that has appeared in previous chapters.

\begin{enumerate}
\item From Chapter~\ref{:rectangles}, can we remove the limitation that $1/2 < \rho$ when querying on arbitrarily-oriented line segments?  
This limitation stems from our current solution relying on the centrepoint of the located segments residing inside the query region.

\item From Chapter~\ref{:rectangles}, can we classify the endpoints of the segments in a way that reduces the number of partial enclosure expressions that we need to test. 
The correct partial enclosure expression depends on which boundaries of the query rectangle are crossed by a segment, but our current classification technique is limited in how well it can identify these.

\item From Chapter~\ref{:rectangles}, in a practical setting, when we are better off saving a constant factor of space by sharing the endpoint classification structures versus saving a $\log{n}$ factor by performing the half-plane queries first.

\item From Chapter~\ref{:slabs}, extending our method to arbitrarily-oriented segments results in a very case-heavy data structure. Moreover, the partial enclosure expressions involve several independent combinations of query variables resulting in high-degree half-plane queries. 
Is there a better method of classification that can reduce this?

\item From Chapter~\ref{:monotonep}, can we modify our approach to building a multi-region formula tree to support querying on simple polygons?  Our tree relies on clean boundaries between regions so that we can query a single formula for any horizontal line query. In a simple polygon setting, two regions may be located one over top of the other without sharing common boundaries.  Attempting to partition the polygon by the boundary of one region can create subregions which break our area formulas.

\end{enumerate}
