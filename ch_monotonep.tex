\chapter{Partial Enclosure Range Searching on Monotone Polygons}
\label{:monotonep}


In this chapter, we show how we can answer the partial enclosure range searching problem on monotone polygons. 
The main contribution of this chapter is a method to calculate the area of a monotone polygon appearing within a query rectangle. 
Once the enclosed area is calculated, determining if the partial enclosure property is satisfied is straight-forward.


\section{Problem Definition}
\label{:monotonep:problem-definition}

The following problem statement describes our goal for this chapter.

\begin{problem}
We are given a monotone polygon $P$ consisting of $n$ vertices and a fixed parameter $\rho$ such that $0 < \rho \leq 1$. We want to determine whether at least $\rho \cdot \area{P}$ is enclosed by an axis-parallel query rectangle $Q$.
\end{problem}

Throughout this chapter, we will use the following definitions.
The input is a polygon $P$ embedded in the plane which, without loss of generalization, is monotone with respect to the $x$-axis (i.e., from left to right). 
$P$ is defined by its vertices $v_1, v_2, \ldots, v_n$, where $v_n$ and $v_1$ share an edge to close the polygon.

A query $Q$ is given by its lower-left and upper-right corners $(\alpha, \beta)$ and $(\gamma, \delta)$, respectively. See Figure~\ref{fig:monotonep:query-example} for an example.

\begin{figure}[t]
\begin{center}
  \includegraphics[width=0.80\textwidth]{figures/fig_mono_query}
  \caption[A monotone polygon $P$ with query $Q$.]{A monotone polygon $P$ with query $Q$. The area of $P$ enclosed by $Q$, $P \cap Q$, is highlighted.}
  \label{fig:monotonep:query-example}
\end{center}
\end{figure}

We say that $P$ is sufficiently enclosed by $Q$ to satisfy the partial enclosure property if and only if $\area{Q \cap P} \geq \rho \cdot \area{P}$.


%------------------------------------------------------------------------------
\section{A First Problem}
\label{:monotonep:approach}

We will first describe an algorithm which solves a simpler problem, and then extend that solution into one for our main problem.

\begin{problem}
Given a monotone polygon $P$, and a query in the form of a horizontal slab $S$, what is the total area of $P$ enclosed in $S$?  i.e., what is $\area{S \cap P}$?
\end{problem}

Our overall approach for this problem is to define a function (actually, a collection of functions) which returns the area below a given horizontal query line. 

% -----------------------------------------------------------------------------
\subsection{Decomposing $P$}
\label{sec:decompose}

In order to create an area formula for the entire polygon, we decompose $P$ into a linear number of regions and calculate area formulas for each. These regional formulas are then combined into an overarching \emph{multi-region formula}.

We begin by sorting all of the vertices by their $x$-coordinates. 
If two vertices share the same $x$-coordinate (no more than two can do so, since $P$ is monotone), we can skip the duplicate without any other change to the algorithm. 
However, for ease of discussion, we will assume that every vertex has a distinct $x$-coordinate.
For the remainder of this chapter, we relabel the $x$-coordinates by their sorted order so that $x_1$ is the $x$-coordinate of the leftmost vertex of $P$, $x_2$ the next leftmost, etc.

Let $\iline{x}$ be the vertical line through any $x$-coordinate, and let $P_U$ and $P_L$ be the upper and lower chains of $P$, respectively. Consider the sequence of lines $\iline{x_1}, \iline{x_2}, \ldots, \iline{x_n}$ through the vertices of $P$.
For every line $\iline{x_i}$, let $a_i = \iline{x_i} \cap P_U$ and $a_i' = \iline{x_i} \cap P_L$ be the intersections of $\iline{x_i}$ with the upper and lower chains of $P$, respectively.

We will calculate all of these intersections as we walk from left to right over both $P_U$ and $P_L$ simultaneously. 
We also keep track of $e_i$ and $e_i'$, the edges of $P_U$ and $P_L$, respectively, which are intersected by $\iline{x_i}$, and upon which reside the points $a_i$ and $a_i'$. 
Where $\iline{x_i}$ intersects a vertex of $P_U$ or $P_L$, we store the edge to the right of that vertex as the corresponding value of $e_i$ or $e_i'$.
Finally, this walk allows us to generate a sequence of regions $R_1, R_2, \dots, R_{n-1}$ which have the following, equivalent definitions.
\begin{itemize}
 \item For each $1 \leq i \leq n - 1$, let $X_i$ be the vertical slab between $\iline{x_i}$ and $\iline{x_{i+1}}$, then $R_i = X_i \cap P$.

 \item $R_i$ is the area of $P$ bounded by the vertical segments $\iline{x_i} \cap P$ and $\iline{x_{i+1}} \cap P$ to the left and right, and some portion of the edges $e_i$ and $e_i'$ to the top and bottom (while, in general, $e_i$ and  $e_i'$ extend beyond $\iline{x_i}$ and $\iline{x_{i+1}}$, this excess is irrelevant to the definition of $R_i$).
 
 \item $R_i$ is the polygon formed by the cycle on $a_i, a_{i+1}, a_{i+1}', a_i'$.
\end{itemize}

\noindent
Thus, each region is a (possibly degenerate) trapezoid.


% -----------------------------------------------------------------------------
\subsection{Area of a Region}
\label{sec:region_area}

For each region $R_i \in \{R_1, R_2, \ldots R_{n-1}\}$, we create a function $F(R_i, h)$ which will return the area of $R_i$ below a horizontal line query line having a $y$-component of $h$.
Figure \ref{fig:monotonep:trapezoid} shows an example region.
From bottom to top, we can identify at most 4 ``critical heights'', where the nature of how the area of a region grows with respect to $h$ changes.
For a general region $R$, these 4 critical heights are:

\begin{figure}[t]
\begin{center}
  \includegraphics[width=0.20\textwidth]{figures/fig_mono_trap}
  \caption[A trapezoidal region of $P$.]{A trapezoidal region of $P$ for which we can derive an area function dependant on $h$.}
  \label{fig:monotonep:trapezoid}
\end{center}
\end{figure}

\begin{itemize}
\item $h_0$, where $R$ begins.
\item $h_1$, where $R$ stops growing as a triangle and begins growing as a rectangle.
\item $h_2$, where $R$ stops growing as a rectangle and begins growing as the base trapezoid of a triangle.
\item $h_3$, where $R$ ends.
\end{itemize}

These 4 critical heights give rise to a piece-wise function representing the area which is defined as follows.
\[
F(R, h) = \begin{cases}
0                  & \text{if }          h < h_0 \\
A(h')^2            & \text{if } h_0 \leq h < h_1 \text{ for } h' = h - h_0 \\
Bh' + C            & \text{if } h_1 \leq h < h_2 \text{ for } h' = h - h_1 \\
A'(h')^2 + C'      & \text{if } h_2 \leq h < h_3 \text{ for } h' = h_3 - h \\
C''                & \text{if } h_3 \leq h \\
\end{cases}
\]

In essence, the area function is a quadratic one, although, for different values of $h$, some or all of the coefficients are 0. The details of each part of this function, and the definitions of the constants $A, A', B, C, C'$ and $C''$ are as follows.

\begin{enumerate}
\item When $h < h_0$, the area of $R$ below $h$ is 0, since no part of $R$ exists below that measure.

\item When $h_0 \leq h < h_1$, the area of $R$ below $h$ grows as a triangle. Let $h' = h - h_0$.
The angle $\theta'$ between $e_i'$ and the vertical can be calculated as 
\[ 
\tan(\theta') = \frac{x_{i+1} - x_i}{h_1 - h_0}
\]

Let $t'$ be this ratio, then at $h'$, the base of the triangle has width $h' \cdot t'$. The area of the triangle is therefore $\frac{1}{2} (h')^2 t'$ and we set the constant $A = \frac{1}{2} t'$.

\item When $h_1 \leq h < h_2$, the area of $R$ includes all of the area between $h_0$ and $h_1$, which we store as the constant $C$.
The remaining area of $R$ up to $h'$ grows as a rectangle.
Let $h' = h - h_1$, then the rectangular area is $(x_{i+1} - x_i)\cdot h'$, and we define the constant $B = (x_{i+1} - x_i)$.
Thus, the total area of $R_i$ up to $h'$ is $Bh' + C$.

\item When $h_2 \leq h < h_3$, the area of $R$ includes all of the area between $h_0$ and $h_2$, which we store as the constant $C_a'$.
The remaining area of $R$ up to $h$ grows as the base trapezoid of a triangle.
Let $h' = h_3 - h$ (note that this definition is somewhat reversed from the other cases).
We can calculate this area by taking the total area of the triangle, and subtracting the area of the triangle from $h_3$ down to $h'$.  

The overall triangle between $h_2$ and $h_3$ has area 
\[ 
a = \frac{1}{2} \cdot (h_3 - h_2) \cdot (x_{i+1} - x_i) 
\]

The angle $\theta$ between $e_i$ and the vertical can be calculated as
\[ 
\tan(\theta) = \frac{x_{i+1} - x_i}{h_3 - h_2}
\]

Let $t$ be this ratio, then the smaller triangle from $h_3$ down to $h'$ can be calculated as in case 2, giving $a' = \frac{1}{2}(h')^2 t$. 
The overall area from $h_2$ up to $h$ is then
\[ 
\begin{split}
a' - a &= \frac{1}{2}(h')^2 t - \frac{1}{2} \cdot (h_2 - h_3) \cdot (x_{i+1} - x_i) \\
%
\end{split}
\]

This expression is of the form $A'(h')^2 + C_b'$.
Thus we set the constants $A' = \frac{1}{2}t$, $C_b' = -\frac{1}{2} \cdot (h_2 - h_3) \cdot (x_{i+1} - x_i)$, and $C' = C_a' + C_b'$ to complete the formula for this part of $F(R, h)$.

\item When $h_3 \leq h$, the total area of $R$ should be reported, which we store in $C''$.

\end{enumerate}

% -----------------------------------------------------------------------------
\subsection{Creating Multi-Region Formulas}
\label{sec:mr_formula}

After completing the previous step, we have a set of regions, and their area formulas, $R_i$ and $F(R_i, h)$ respectively for $1 \leq i \leq n-1$.

To complete the preprocessing for our horizontal slab query, we will process our regional area formulas into a list of multi-region formulas which can report the area of the entire polygon under a query line.

We begin by collecting tuples of all of the critical heights from our regional formulas. 
For every region $R_i$, we collect $(y, R)$ for $y \in \{h_0, h_1, h_2, h_3\}$ and $R = R_i$.
Let $\mathcal{Y}$ be the list of all such tuples across all regions, sorted by $y$-values from lowest to highest.

If we suppose for a moment that each $y$-value is distinct, then for each one there is exactly one region $R$ which is transitioning from one phase of growth to another.
That is, looking at the piecewise function $F(R, h)$ which determines the area of $R$ below a query line, a new piece of that function is taking over.
To create the multi-region formula for a new critical height, we copy the formula for the predecessor height (the lowest $y$-value starts with 0).
To update the multi-region formula, we subtract the coefficients of $R$ for its previous phase of growth so that they no longer influence the formula, and then add the coefficients for the new phase.

In practice, this algorithm does not require the values of $y$ to be distinct. Instead, several values of $y$ will be collapsed into a single critical height in the list of multi-region formulas. Algorithm \ref{alg:monotonep:mf} gives the details of this process more formally.

\begin{algorithm}
\LinesNumbered
\DontPrintSemicolon
\caption{BuildMultiRegionFormula}
\label{alg:monotonep:mf}
\SetKwFunction{fSort}{sort}
\KwIn{List of regions $\mathcal{R} = R_1, R_2, \ldots, R_{n-1}$}
\BlankLine
Initialize a list $\mathcal{Y}$\;
\ForEach{$R$ in $\mathcal{R}$}
{
$\mathcal{Y} \gets \mathcal{Y} \cup \{ (y,R) \;|\; y \in \{R.h_0, R.h_1, R.h_2, R.h_3\}\}$\;
}
\fSort{$\mathcal{Y}$ on $y$}\;
\BlankLine
$y' \gets $ minimum $y$-value in $\mathcal{Y}$\;
$F(P, y') \gets $ coefficients $A,B,C$ set to 0\;
\BlankLine
\ForEach{$y, R$ in $\mathcal{Y}$}
{
$F(P, y) = F(P, y')$\;
$i \gets \{0,1,2,3\}$ such that $y = h_i \in \{R.h_0, R.h_1, R.h_2, R.h_3\}$\;
\If{$i > 0$}{
$F(P,y).A \gets F(P,y).A - R.h_{i-1}.A$\;
$F(P,y).B \gets F(P,y).B - R.h_{i-1}.B$\;
$F(P,y).C \gets F(P,y).C - R.h_{i-1}.C$\;
}
$F(P,y).A \gets F(P,y).A + R.h_i.A$\;
$F(P,y).B \gets F(P,y).B + R.h_i.B$\;
$F(P,y).C \gets F(P,y).C + R.h_i.C$\;
\BlankLine
$y' \gets y$\;
}
\BlankLine
$H \gets $ the list of all $F(P, h)$ created above\;
\KwRet{$H$}
\end{algorithm}

The output of this algorithm is the list $H$ of multi-region formulas. Since $\mathcal{Y}$ was sorted by $y$-value, $H$ will be as well.


% -----------------------------------------------------------------------------
\subsection{Querying Multi-Region Formulas}
\label{:monotonep:query-mf}

We are now ready to query horizontal slabs against $P$. 
Let $F(P,h)$ be the multi-region area function for $P$, which is essentially a very big piecewise function given by Algorithm \ref{alg:monotonep:mf} in the form of a sorted list $H$.
Given two horizontal lines defining our query slab, located at $\beta$ and $\delta$, $\beta \leq \delta$, our query requires only a few steps.

\begin{enumerate}
\item Our query parameter $\beta$ will not generally correspond to a critical height in $H$.
Perform a binary search on $H$ to find the formula stored with the \emph{predecessor} of $\beta$.
Denote this formula as $F_\beta$. 
Evaluating $F_\beta(P, \beta)$ tells us the area of $P$ below $\beta$.

\item For $\delta$, we perform another binary search on $H$ to find $F_\delta$ such that $F_\delta(P, \delta)$ tells us the area of $P$ below $\delta$.

\item We can now calculate the area inside the query slab as
\[ 
\area{S \cap P} = F_\delta(P, \delta) - F_\beta(P, \beta)
\]

\end{enumerate}

\noindent
We summarise our results with the following theorem and corollary.

\begin{theorem}
\label{th:montonep:slab-area}
Let $P$ be a monotone polygon consisting of $n$ vertices. In $\BigOh{n \log{n}}$ time and $\BigOh{n}$ space, we can create a data structure that allows us to determine $\area{S \cap P}$ in $\BigOh{\log{n}}$ time for any horizontal slab query region $S$.
\end{theorem}

\begin{corollary}
\label{cor:montonep:slab-mp}
Let $P$ be a monotone polygon consisting of $n$ vertices, and let $0 < \rho \leq 1$ be a fixed parameter. 
With $\BigOh{n \log{n}}$ preprocessing time and $\BigOh{n}$ space, we can determine if $\area{S \cap P} \geq \rho \cdot \area{P}$ in $\BigOh{\log{n}}$ time for any horizontal slab $S$.
\end{corollary}


% -----------------------------------------------------------------------------
% -----------------------------------------------------------------------------
\section{Extending to Rectangular Queries}
\label{:monotonep:rect}

Our actual query, as introduced in Figure \ref{fig:monotonep:query-example}, is a rectangular area. Our solution to this type of query is extended from the methods we used to solve the horizontal slab queries.

\subsection{Preprocessing}
\label{:monotonep:rect:pre}

We develop a list of regions $\mathcal{R} = R_1, R_2, \ldots, R_{n-1}$ just as in Section \ref{sec:decompose}.  Now, instead of constructing a single list of multi-region formulas to cover all of $P$, we will construct a tree of such lists so that for any $1 \leq i \leq j \leq n-1$, we can query the subpolygon $\bigcup_{k=i}^{j}{R_k}$ for its area below a query line $h$ in $\BigOh{\log{n}}$ time.

Let this multi-region formula tree be $T$; we construct $T$ in the following way. First, construct the multi-region formula tree for each half of the regions recursively, giving the subtrees $T_l$ and $T_r$. If $|\mathcal{R}| = 1$ for any recursive step, we create a leaf and return the area formula for that single region.

Let $H(T_l)$ and $H(T_r)$ be the multi-region formula lists for $T_l$ and $T_r$, respectively. 
We need to create a merged list, $H(T)$ representing the regions of both subtrees together. 
Unfortunately, we cannot just naively merge the formulas into a combined sorted list as each formula is only concerned with the regions upon which it was originally created. 

Instead, we need to generate new coefficients for each critical height of $h$ which will be valid for all regions of the combined area.
This process is precisely Algorithm \ref{alg:monotonep:mf} \emph{without} the initial sorting step, which we can avoid since $H(T_l)$ and $H(T_r)$ are already sorted.
Thus, we can build $H(T)$ in time $\BigOh{|H(T_l)| + |H(T_r)|}$.
Algorithm \ref{alg:monotonep:bmft} gives more formal details.

\begin{algorithm}
\LinesNumbered
\DontPrintSemicolon
\caption{BuildMultiRegionFormulaTree}
\label{alg:monotonep:bmft}
\SetKwFunction{fSort}{sort}
\SetKwFunction{fBMFT}{BuildMultiRegionFormulaTree}
\KwIn{List of regions $\mathcal{R} = R_1, R_2, \ldots, R_{n-1}$}
\BlankLine
\If{$|\mathcal{R}| = 1$}{
$T \gets $ create new leaf node\;
$H(T) \gets F(R,h)$\;
\KwRet{$T, H(T)$}
}
$m \gets \floor{\frac{|\mathcal{R}|}{2}}$\;
$\mathcal{R}_l = \{ R_1, R_2, \ldots, R_m \}$\;
$\mathcal{R}_r = \{ R_{m+1}, R_{m+2}, \ldots, R_{n-1} \}$\;
$T_l, H(T_l) \gets $ \fBMFT{$R_l$}\;
$T_r, H(T_r) \gets $ \fBMFT{$R_r$}\;
$T \gets $ Create new node with left child $T_l$ and right child $T_r$\;
$H(T) \gets $ Merge $H(T_l)$ and $H(T_r)$ using Algorithm~\ref{alg:monotonep:mf}\;
\BlankLine
\KwRet{$T, H(T)$}
\end{algorithm}

In the last step of the algorithm we create a list of formulas over all regions, which implies that we can answer horizontal slab queries from the root of $T$. Recursing on half of the regions at each step gives us a tree which will be balanced with depth $\BigOh{\log{n}}$. At each level of $T$, every critical height for every region is considered, resulting in $\BigOh{n}$ area formulas in the worst case where every critical height is distinct. As mentioned, the merging step at each node is completed in linear time with respect to the number of regions processed. Therefore, the total time and storage required to build $T$ is $\BigOh{n\log{n}}$.  


% -----------------------------------------------------------------------------
\subsection{Querying}
\label{:monotonep:rect:querying}

\begin{figure}[t]
\begin{center}
  \includegraphics[width=0.80\textwidth]{figures/fig_mono_extra}
  \caption[Details of a query on a monotone polygon.]{A monotone polygon $P$ with query $Q$. The tiled areas cannot be queried using preprocessed area formulas and will require special handling.}
  \label{fig:monotonep:extra}
\end{center}
\end{figure}

Our query rectangle $Q$ is given by the lower-left and upper-right coordinates $(\alpha, \beta)$ and $(\gamma, \delta)$, respectively. We define $h_\beta$ and $h_\delta$ as the horizontal lines through $\beta$ and $\delta$, respectively, which we can use as inputs to our region area formulas.

Let $\iline{\alpha}$ and $\iline{\gamma}$ be vertical lines through $\alpha$ and $\gamma$, respectively.
Using a binary search on the $x$-values of $P$, we can identify the regions which $\iline{\alpha}$ and $\iline{\gamma}$ pass through.
Let $V(Q)$ be the vertical slab defined by $Q$; that is, the vertical area between $\iline{\alpha}$ and $\iline{\gamma}$.
We can calculate the area of $Q \cap P$ by considering how $Q$ interacts with the following areas:

\begin{enumerate}
 \item The leftmost region, where $R_i \not \subset V(Q)$. Let $A_i = \area{Q \cap R_i}$.
 \item The center regions $R_{i+1}, R_{i+2}, \ldots, R_{j-1}$, where  $R_k \subset V(Q)$ for $i + 1 \leq k \leq j -1$. Let $A_c = \area{Q \cap \left ( \bigcup_{k=i+1}^{j-1}{R_k} \right )}$.
 \item The rightmost region, where $R_j \not \subset V(Q)$. Let $A_j = \area{Q \cap R_j}$.
\end{enumerate}

See Figure \ref{fig:monotonep:extra} for an example. We begin with the center regions. All of these regions are entirely within $V(Q)$, and so we can use their precalculated area formulas directly. That is:

\[
A_c
= \area{Q \cap \left ( \bigcup_{k=i+1}^{j-1}{R_k} \right )}
= \sum_{k=i+1}^{j-1} F(R_k, h_\delta) - F(R_k, h_\beta)
\]

\noindent Performing this sum naively takes $\BigOh{n}$ time as there may be a linear number of regions spanned by $Q$.
However, using $T$, we can answer this query for any values of $i$ and $j$ by checking at most $\BigOh{\log{n}}$ subtrees. 
At each subtree, we require $\BigOh{\log{n}}$ time to find the correct formula, for a total query time of $\BigOh{\log^2{n}}$.
However, since each subtree queries for the same value of $h$, we can reduce this query time to only $\BigOh{\log{n}}$ by using fractional cascading.\cite{cg-fc-86, cg-fc2-86}

Considering $A_i$ now, if $\alpha = x_i$, then $A_i = F(R_i, h_\delta) - F(R_i, h_\beta)$. 
In general, however, $x_i < \alpha < x_{i+1}$, and we cannot use our precalculated area formulas.
Fortunately, $Q \cap R_i$ is a polygon of $\BigOh{1}$ complexity, specifically a trapezoid, so its area can be calculated directly in constant time. 
Recall from the construction of the list of regions, $\mathcal{R}$, that if $\iline{\alpha}$ passes between $x_i$ and $x_{i+1}$, then it passes through region $R_i$, which stores the edges $e_i$ and $e_i'$ defining its top and bottom. 
We can calculate the intersection points with $\iline{\alpha}$ as $s = e_i \cap \iline{\alpha}$ and $s' = e_i' \cap \iline{\alpha}$.
Recall that $a_{i+1} = \iline{x_{i+1}} \cap e_{i+1}$, and $a_{i+1}' = \iline{x_{i+1}} \cap e_{i+1}'$.
Thus, the vertices of $Q \cap R_i$ are:
\begin{itemize}
 \item Its top-left: the lower of $(\alpha, s)$ and $(\alpha, \delta)$,

 \item Its bottom-left: the higher of $(\alpha, s')$ and $(\alpha, \beta)$,

 \item Its top-right: the lower of $a_{i+1}$ and $h_\delta \cap \iline{x_{i+1}}$, and

 \item Its bottom-right: the higher of $a_{i+1}'$ and $h_\beta \cap \iline{x_{i+1}}$.

\end{itemize}

Likewise, if $\gamma = x_j$, then $A_j = 0$. But, in general, $x_j < \gamma < x_{j+1}$, and, again, we cannot use our precalculated area formulas.
$Q \cap R_j$ is also a trapezoid, however, so its area can be calculated directly in a similar way to $A_i$.

With all three area calculations completed, a simple sum completes our query. We summarize our results with the following theorem and corollary.

\begin{theorem}
\label{th:monotonep:rect:area}
Let $P$ be a monotone polygon consisting of $n$ vertices. 
In $\BigOh{n \log{n}}$ time and $\BigOh{n \log{n}}$ space, we can create a data structure which allows us to determine $\area{Q \cap P}$ in $\BigOh{\log{n}}$ time, for any axis-parallel rectangular query region $Q$.
\end{theorem}

\begin{corollary}
\label{cor:monotonep:rect:mp}
Let $P$ be a monotone polygon consisting of $n$ vertices, and let $0 < \rho \leq 1$ be a fixed parameter.
With $\BigOh{n \log{n}}$ preprocessing time and $\BigOh{n \log{n}}$ space, we can determine if $\area{Q \cap P} \geq \rho \cdot \area{P}$ in $\BigOh{\log{n}}$ time for any axis-parallel rectangular query $Q$.
\end{corollary}


%------------------------------------------------------------------------------
%------------------------------------------------------------------------------
\section{An Alternate Approach}
\label{:monotonep:alt}

In this section, we detail an alternate approach to solving this problem which sacrifices some query time in favour of lowering our space requirements to only $\BigOh{n}$.

We begin by decomposing $P$ into $\BigOh{n}$ trapezoidal regions and creating area formulas for each just as in Section~\ref{:monotonep:approach}. 
Next, we divide $P$ into $\BigOh{\sqrt{n}}$ multi-regions, from left to right, each comprised of $\BigOh{\sqrt{n}}$ trapezoidal regions.
For each multi-region, we calculate its multi-region area formula using Algorithm~\ref{alg:monotonep:mf}.  
Altogether, this process requires $\BigOh{n\log{n}}$ time, and $\BigOh{n}$ storage.

Our query $Q$ is a box which is open to the left and bottom.
We can specify $Q$ by a single point with coordinates $(\gamma, \delta)$.  
Figure~\ref{fig:mono2:query} shows an example query and a decomposition of $P$ into $\BigOh{\sqrt{n}}$ multi-regions.

\begin{figure}[t]
\begin{center}
  \includegraphics[width=0.80\textwidth]{figures/fig_mono2_query}
  \caption[An alternate query method for a monotone polygon.]{A monotone polygon $P$ with query $Q$. The query region is box which is open to the left and bottom.}
  \label{fig:mono2:query}
\end{center}
\end{figure}

Performing our query requires only a few steps which consider successively finer regions of our decomposition of $P$.
With a binary search, we can identify the multi-regions $S_1, S_2, \ldots, S_{k-1}$, where $1 \leq k < \sqrt{n}$, which are \emph{entirely} to the left of $\gamma$, and at most one multi-region, $S_{k}$, which is intersected by $\gamma$.  
For each $S_i$, $1 \leq i \leq k-1$, we evaluate the associated multi-region area formula with respect to $\delta$. 
This requires $\BigOh{\log{|S_i|}} = \BigOh{\log{n}}$ time per multi-region for a total of $\BigOh{\sqrt{n}\log{n}}$ time.

We then consider $S_k$, which is intersected by $\gamma$.  
$S_k$ is comprised of $\BigOh{\sqrt{n}}$ trapezoidal regions, some of which are entirely left of $\gamma$, and at most one of which is intersected by $\gamma$. 
For all of those regions entirely left of $\gamma$, we evaluate the regional area formulas associated with them with respect to $\delta$. 
The intersection of the final, partial trapezoidal region which is intersected by $\gamma$ has only $\BigOh{1}$ complexity, so we can evaluate its area directly.

This completes the query. In total, we require the following time to calculate the area inside of $Q$:
\[ 
\BigOh{\log{n}} + \BigOh{\sqrt{n}\log{n}} + \BigOh{\sqrt{n}} + \BigOh{1} = \BigOh{\sqrt{n}\log{n}}
\]

\noindent We can extend this method to query on a closed rectangle by combining the results of four separate open box queries.
Let $Q'$ be a closed rectangle defined by its lower-left point $(\alpha, \beta)$ and its upper-right point $(\gamma, \delta)$.
We can find $\area{Q'}$ in the following way:

\begin{itemize}
\item Let $Q_1$ be the open query box on $(\gamma, \delta)$.
\item Let $Q_2$ be the open query box on $(\gamma, \beta)$, which captures excess area under $Q'$.
\item Let $Q_3$ be the open query box on $(\alpha, \delta)$, which captures excess area left of $Q'$.
\item Let $Q_4$ be the open query box on $(\alpha, \beta)$, which captures the area which is double counted by $Q_2$ and $Q_3$.
\end{itemize}

\noindent Then,
\[
\area{Q'} = \area{Q_1} - \area{Q_2} - \area{Q_3} + \area{Q_4}
\]

%------------------------------------------------------------------------------
\subsection{Improving Query Time}
\label{:mono2:bitfaster}

We can reduce the query time from the above method to only $\BigOh{\sqrt{n}}$ by grouping the regions into three different sized groups instead of just two.

As before, we decompose $P$ into $\BigOh{n}$ trapezoidal regions and create their area formulas.
Next, from left to right, we group the regions into $\frac{\sqrt{n}}{\log{n}}$ multi-regions each containing $\BigOh{\sqrt{n}\log{n}}$ trapezoidal regions.
We refer to these as the A-level regions, and calculate a multi-region formula for each using Algorithm~\ref{alg:monotonep:mf}.
Total preprocessing time and space for the A-level regions is $\BigOh{n\log{n}}$ and $\BigOh{n}$, respectively.
Within each A-level region, we create B-level multi-regions consisting of $\BigOh{\sqrt{n}}$ trapezoidal regions each, again from left to right, resulting in $\BigOh{\log{n}}$ B-level regions for each A-level region.
Total preprocessing time and space for the B-level regions is also $\BigOh{n\log{n}}$ and $\BigOh{n}$, respectively, across all A-levels regions.

Our queries use the same definitions as from Section~\ref{:monotonep:alt}. 
Given a query $Q$ defined by $(\gamma, \delta)$, we first consider what areas are left of $\gamma$, and then use $\delta$ as the input to our area formulas.

We start by considering the A-level regions which are entirely left of $\gamma$. 
Each such region contains $\BigOh{\sqrt{n}\log{n}}$ trapezoidal regions, so querying the multi-region area formula stored there will require the following time.
\[
\log{\BigOh{\sqrt{n}\log{n}}} = \BigOh{\log{n}}
\]

\noindent Therefore, the worst case time required to query all necessary A-level regions is:
\[
\frac{\sqrt{n}}{\log{n}} \cdot \BigOh{\log{n}} = \BigOh{\sqrt{n}}
\]

\noindent At most one A-level region is intersected by $\gamma$, and if that is the case, we proceed to its respective B-level regions.
Of these B-level regions, at most $\BigOh{\log{n}}$ are entirely to the left of $\gamma$.
Each has size $\BigOh{\sqrt{n}}$, so querying the multi-region area formulas stored with them requires $\BigOh{\log{n}}$ time each. Thus, the total query time for B-level regions is $\BigOh{\log^2{n}}$.

At the last level, at most one B-level region is intersected by $\gamma$. 
This region contains $\BigOh{\sqrt{n}}$ trapezoidal regions which are entirely left of $\gamma$.
We can evaluate the area formula stored with each region in $\BigOh{\sqrt{n}}$ total time.
Finally, there may be one trapezoidal region which is intersected by $\gamma$.
This intersection produces an area of $\BigOh{1}$ complexity which we can calculate directly.

Total time required for this query is as follows.
\[
\BigOh{\log{n}} + \BigOh{\sqrt{n}} + \BigOh{\log^2{n}} + \BigOh{\sqrt{n}} + \BigOh{1} = \BigOh{\sqrt{n}}
\]

\noindent Total preprocessing time is $\BigOh{n \log{n}}$ time for sorting and building region and multi-region area formulas.  
Total space required is $\BigOh{n}$ as each trapezoidal region is considered in only one A-level region and in only one B-level region. 
We summarize these results with the following theorem.

\begin{theorem}
\label{th:mono2}
Let $P$ be a monotone polygon consisting of $n$ vertices. 
In $\BigOh{n \log{n}}$ time and $\BigOh{n}$ space, we can create a data structure which allows us to determine $\area{Q \cap P}$ in $\BigOh{\sqrt{n}}$ time, for any axis-parallel rectangular query region $Q$.
\end{theorem}


% -----------------------------------------------------------------------------
% -----------------------------------------------------------------------------
\subsection{Remarks on Simple Polygons}
\label{:monotonep:simplep}

We can apply our method for horizontal slab queries ``as is'' to simple polygons since nothing about the way that we decompose $P$, build the multi-region area formulas, or query them, depends on the monotone property.  Put another way, we do not require any special handling for regions which appear above or below other regions.

To answer slab queries on simple polygons, we decompose $P$ by extending rays from each vertex into the interior of $P$ until they strike the next boundary.  The regions will have the same 4-sided shape as in the monotone polygon case, and we can create an area formula for each one over the, at most, four critical heights of $h$.

Each region still maintains its own critical heights for $h$, and we use Algorithm \ref{alg:monotonep:mf} without modification, first sorting all critical points together, then maintaining a set of coefficients which we adjust by region as we sweep from bottom to top.

Finally, querying the resulting list of multi-region formulas, $H$, is done using the same binary search method as in the monotone case. This result is summarized in the following corollary.

\begin{corollary}
\label{cor:monotonep:simplep-area}
Let $P$ be a simple polygon consisting of $n$ vertices.
In $\BigOh{n \log{n}}$ time and $\BigOh{n}$ space, we can create a data structure which allows us to determine $\area{S \cap P}$ in $\BigOh{\log{n}}$ time for any horizontal slab query region $S$.
\end{corollary}

Unfortunately, we cannot extend our method for rectangular queries to work with simple polygons so easily.
While the multi-region formulas themselves do not use the monotone property, our tree of multi-region formulas does. 
The tree functions by partitioning the trapezoidal regions with respect to vertical lines, however, in a simple polygon, a vertical line passing through the boundary of one region may pass through the interior of another. 
This lack of clean partitioning prevents the multi-region formula from working correctly for all possible horizontal query lines which may be given as input to the formula.

%------------------------------------------------------------------------------
%------------------------------------------------------------------------------
\section{Conclusion}
\label{:monotonep:concl}

In this chapter we have developed methods for calculating the area of a monotone polygon enclosed by a query slab or rectangle.  Specifically, we have described:

\begin{enumerate}
\item A method for calculating the area of a monotone polygon enclosed in a horizontal slab, Theorem~\ref{th:montonep:slab-area}.

\item A method for calculating the area of a monotone polygon enclosed in an axis-parallel rectangle, Theorem~\ref{th:monotonep:rect:area}.

\item An alternate method for calculating the area of a monotone polygon enclosed in an axis-parallel rectangle which is more space efficient, but requires more query time; Theorem~\ref{th:mono2}.

\item A method for calculating the area of a simple polygon enclosed in a horizontal slab, Corollary~\ref{cor:monotonep:simplep-area}.

\end{enumerate}

In every case, once the enclosed area is known, deciding whether the partial enclosure property is satisfied is just a matter of testing a simple inequality.

One possible direction for future research is to consider our options for extending Theorem~\ref{th:monotonep:rect:area} to simple polygons. 
We've already seen that it does not work directly, but there may be alternate methods for partitioning the multi-region formulas or accounting for partially enclosed regions.
