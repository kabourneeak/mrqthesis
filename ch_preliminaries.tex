\chapter{Preliminaries}
\label{:prelim}

In this chapter, we outline the data structures and other techniques which we rely on for solutions to our own contributions.

%------------------------------------------------------------------------------
\section{Range Trees}
\label{:prelim:range-trees}

A range tree is a binary search tree that allows us to search for elements satisfying a 1D query interval.
Given a set of $n$ elements, $A = \{a_1, a_2, \ldots, a_n\}$, each expressing a scalar ``key'' value $v_1, v_2, \ldots, v_n$, a range tree $T$ can identify the elements whose keys fall between two values $\alpha$ and $\beta$, where $\alpha \leq \beta$.
That is, the search identifies the following set:
\[
\{ a_i \in A \st \alpha \leq v_i \leq \beta \}
\]

This search can be done in $\BigOh{\log{n}}$ time with only $\BigOh{n}$ space and $\BigOh{n\log{n}}$ preprocessing time.
Reporting the matching elements can be done in $\BigOh{\log{n} + k}$ time, where $k$ is the number of elements reported. 

However, all of these properties are also true of a regular binary search on a sorted array, so why are range trees interesting? 
To answer this, we look at a simple example.
Suppose $A$ is a set of 2D points embedded in the plane, and that, for our query, we wish to identify all points whose $x$-coordinate is between $\alpha$ and $\beta$.
To answer this query, we can construct a range tree $T$ where each point $a_i$, $1 \leq i \leq n$ is represented by leaf node whose key value is set to the $x$-coordinate of $a_i$; that is, where $v_i = a_i.x$.

When we query $T$ for $\alpha$ and $\beta$, we identify two paths through $T$ in $\BigOh{\log{n}}$ time. The leaf with key value $\alpha$ (or with the successor value to $\alpha$, if the exact value is not present) is $a_\alpha$ and is the leftmost point in $A$ matching our query.  Similarly, the leaf with key value $\beta$ (or with the  predecessor to $\beta$, if the exact value is not present) is $a_\beta$ and is the rightmost point in $A$ matching our query. Since the leaves in a binary search tree are sorted, the sequence of points from $a_\alpha$ to $a_\beta$ are exactly the points we need to identify. We can report these points with an inorder traversal.

\begin{figure}
\begin{center}
  \includegraphics[width=0.40\textwidth]{figures/fig_pre_range1d}
  \caption{A 1D range tree and the subtrees identified by a query.}
  \label{fig:pre:range1d}
\end{center}
\end{figure}

What makes range trees such versatile structures is not that they identify appropriate elements, but rather due to the way that they group the identified elements into at most $\BigOh{\log{n}}$ disjoint subsets.
These subsets are precisely the elements found in the ``inner'' subtrees along the path from $a_\alpha$ to $a_\beta$, as illustrated in Figure~\ref{fig:pre:range1d}.

With each non-leaf node of $T$, we can store additional information about the subset represented by its subtree.  The, during query time, we only need to consider the information stored at $\BigOh{\log{n}}$ nodes rather than the $\BigOh{n}$ elements they represent. Continuing our example on 2D points, with only $\BigOh{1}$ extra space per node, we could store which point has the highest $y$-coordinate in each subtree.  Then, we can use our range tree to find the highest point between two $x$-coordinates in $\BigOh{\log{n}}$ time.

We are not limited to just storing simple values with each node, however.
With each node, we can associate an entire additional data structure, constructed on the elements represented by that node.
In particular, we can store another range tree which is based on some other property of each element.
When we query this two-level range tree, our query at the first level identifies $\BigOh{\log{n}}$ subsets.
Each subset represents elements which satisfy a particular query condition.
Then, for each subset, we query the associated data structure to identify elements satisfying a second query condition.
The elements identified here satisfy both conditions.
This technique can be repeated to create \emph{multi-level range trees} with several levels.

The following theorems on range trees, restated from \cite{debergch5}, detail the construction and query times of multi-level range trees.

\begin{theorem}
\label{th:rangetree}
Let $P$ be a set of $n$ points in $d$-dimensional space, with $d \geq 2$. A multi-level range tree for $P$ uses $\BigOh{n \log^{d-1}{n}}$ storage and it can be constructed in $\BigOh{n \log^{d-1}{n}}$ time. With this range tree one can report the points in $P$ that lie in a rectangular query range in $\BigOh{\log^{d-1}{n} + k}$ time where $k$ is the number of reported points.
\end{theorem}

If each node of the range tree is augmented with information about the size of its subtree, then we can count the number of items within a query rectangle rather than reporting them. The following corollary summarizes this.

\begin{corollary}
\label{cor:rangetree}
Let $P$ be a set of $n$ points in $d$-dimensional space, with $d \geq 2$. A multi-level range tree for $P$ uses $\BigOh{n \log^{d-1}{n}}$ storage and it can be constructed in $\BigOh{n \log^{d-1}{n}}$ time. With this range tree one can count the points in $P$ that lie in a rectangular query range in $\BigOh{\log^{d-1}{n}}$ time.
\end{corollary}


%------------------------------------------------------------------------------
\section{Canonical Subsets Data Structure}
\label{:prelim:chan}

The range trees introduced in the previous section can be used to query any expression comprised of a single query variable.  In some cases, however, our expressions will take the form of half-planes comprised of two query variables, and for those, we use the following structure described in Chan\cite{chan2012}, restated here with $d=2$.

\begin{theorem}[Corollary 7.3, part $i$, in Chan\cite{chan2012}]
\label{th:chan}
We can form $\BigOh{n}$ canonical subsets of total size $\BigOh{n \log{n}}$ in $\BigOh{n \log{n}}$ time, such that the subset of all points inside any query simplex can be reported as a union of disjoint canonical subsets $C_i$ with $\sum_i{\sqrt{|C_i|}} \leq \BigOh{\sqrt{n}\log{n}}$ in time $\BigOh{\sqrt{n}\log{n}}$ w.h.p. $(n)$.
\end{theorem}

This structure is particularly well-suited to multi-part queries where we need to identify objects which satisfy a half-plane query, and which also satisfy some other arbitrary query condition for which we have an existing query method.
To support such a query, we construct a canonical subsets structure to perform the half-plane component of the query. 
Then, with the objects of each canonical subset, we construct a secondary query structure.
A similar structure with slightly worse query time bounds is presented in \citep{debergch16}, along with a method for using it to construct multi-part queries.

A multi-level query is executed by first querying the canonical subsets structure for all points satisfying the half-plane.
With each subset identified by the first step, we then evaluate the associated secondary structure.
The result is the set of all objects which satisfy both the half-plane query \emph{and} whatever conditions the second-level query may test.
If the secondary structure is itself a multi-level structure, then this procedure effectively adds one more layer, and we can repeat this process to any arbitrary number of levels.
The following corollary formalizes the use of canonical subset structures to build multi-level structures and details the preprocessing and query requirements.


\begin{corollary}
\label{cor:multichan}

Given a set of $n$ objects $a_1, a_2, \ldots, a_n$ which can be queried with a data structure $A$ requiring preprocessing space $S(n)$, preprocessing time $T(n)$, and query time $Q(n)$, and where each $a_i$ also has an associated point $p_i$, we can construct a multi-level data structure which can identify all objects whose associated point $p_j$ satisfies a half-plane  \emph{and} where $a_j$ satisfies a query on $A$.

\begin{enumerate}
\item If $S(n) \in \BigOh{n \log^f{n}}$, $T(n) \in \BigOh{n \log^g{n}}$, and $Q(n) \in \BigOh{\sqrt{n} \log^h{n}}$, where $f, g, h \in \BigOh{1}, 0 \leq f \leq g$, then the resulting multi-level data structure requires $\BigOh{n\log^{f+1}{n}}$ preprocessing space, $\BigOh{n\log^{g+1}{n}}$ preprocessing time, and $\BigOh{\sqrt{n}\log^{h+1}{n}}$ query time.

\item If $S(n) \in \BigOh{n \log^f{n}}$, $T(n) \in \BigOh{n \log^g{n}}$, and $Q(n) \in \BigOh{\log^h{n}}$, where $f, g, h \in \BigOh{1}, 0 \leq f \leq g$, then the resulting multi-level data structure requires $\BigOh{n\log^{f+1}{n}}$ preprocessing space, $\BigOh{n\log^{g+1}{n}}$ preprocessing time, and $\BigOh{\sqrt{n}\log^{h+1}{n}}$ query time.

\end{enumerate}
\end{corollary}

\begin{proof}
The preprocessing space of the multi-level structure requires $\BigOh{n \log{n}}$ space for the canonical subsets structure itself, plus
\[
\begin{split}
\sum_{i=1}^k{S(|C_i|)}
&\leq \sum_{i=1}^k{|C_i| \log^f{|C_i|}} \\
%
&\leq \sum_{i=1}^k{|C_i| \log^f{n}} \\
%
&\leq \BigOh{\log^f{n}} \cdot \sum_{i=1}^k{|C_i|} \\
%
&\leq \BigOh{n\log^{f+1}{n}}
\end{split}
\]

\noindent for the associate structures, for a total space complexity of $\BigOh{n\log^{f+1}{n}}$.

Preprocessing time is calculated in the same manner. 
We start with $\BigOh{n \log{n}}$ time for building the canonical subsets structure itself, plus
\[
\begin{split}
\sum_{i=1}^k{T(|C_i|)}
&\leq \sum_{i=1}^k{|C_i| \log^g{|C_i|}} \\
%
&\leq \sum_{i=1}^k{|C_i| \log^g{n}} \\
%
&\leq \BigOh{\log^g{n}} \cdot \sum_{i=1}^k{|C_i|} \\
%
&\leq \BigOh{n\log^{g+1}{n}}
\end{split}
\]

\noindent for building the associate structures, for a total time complexity of $\BigOh{n\log^{g+1}{n}}$.

Querying requires $\BigOh{\sqrt{n}\log{n}}$ time to find the $k'$ disjoint canonical subsets representing the elements found by the top-level canonical subsets query, plus the time required to query the associated data structures.  If $Q(n) \in \BigOh{\sqrt{n}\log^h{n}}$ then the total time to query the appropriate associated is
\[
\begin{split}
\sum_{i=1}^{k'}{Q(|C_i|)}
&\leq \sum_{i=1}^{k'}{\BigOh{\sqrt{|C_i|}\log^h{|C_i|}}} \\
%
&\leq \sum_{i=1}^{k'}{\BigOh{\sqrt{|C_i|}\log^h{n}}} \\
%
&\leq \BigOh{\log^h{n}} \cdot \sum_{i=1}^{k'}{\BigOh{\sqrt{|C_i|}}} \\
%
&\leq \BigOh{\sqrt{n}\log^{h+1}{n}} \\
\end{split}
\]

\noindent Otherwise, if $Q(n) \in \BigOh{\log^h{n}}$ then the total time to query the appropriate associated is
\[
\begin{split}
\sum_{i=1}^{k'}{Q(|C_i|)}
&\leq \sum_{i=1}^{k'}{\BigOh{\log^h{|C_i|}}} \\
%
&\leq \sum_{i=1}^{k'}{\BigOh{\log^h{n}}} \\
%
&\leq \BigOh{\log^h{n}} \cdot \sum_{i=1}^{k'}{1} \\
%
&\leq \BigOh{\log^h{n}} \cdot \BigOh{\sqrt{n}\log{n}}\\
%
&\leq \BigOh{\sqrt{n}\log^{h+1}{n}} \\
\end{split}
\]

\noindent Thus, the query time for the associated structures is identical for both complexity classes of $Q(n)$ that we consider. 
Total query time is $\BigOh{\sqrt{n}\log^{h+1}{n}}$.
\end{proof}

We give an example of how we can use this structure. 
Suppose we have $n$ line segments in the plane.
Each line segment $s_i$, $1 \leq i \leq n$ is given by its left endpoint $p_i = (a_i, b_i)$ and its right endpoint $q_i = (c_i, d_i)$.
We would like to identify all segments intersected by a query line $L: y = \alpha x + \beta$.

We will solve this query in two steps.
First, we identify all segments whose left endpoint is left of $L$.
Of those, we will then identify which have their right endpoint right of $L$.
Describing the construction of a multi-level data structure is often clearer when we begin with the innermost level.
With that in mind, we consider the right endpoints first. 
Considering only the points $q_i$, $1 \leq i \leq n$, and using $L$ to define a half-plane, we apply 
Theorem~\ref{th:chan} to create a structure in $\BigOh{n\log{n}}$ time and space which will identify all points inside the half-plane representing the right side of $L$ in $\BigOh{\sqrt{n}\log{n}}$ time.  
While each subset in this structure is queried by the values of the right endpoints, we can store the entire line segment.

Considering the left endpoints now, applying Theorem~\ref{th:chan} again gives us another structure with identical preprocessing and query times. 
On this structure, we associate an instance of the structure from step 1 with every subset, constructed on only the elements of its respective subset.
By Corollary~\ref{cor:multichan}, this multi-level data structure will require $\BigOh{n\log^2{n}}$ preprocessing time and space, and will identify all segments whose left endpoint is left of $L$ \emph{and} whose right endpoint is right of $L$ in $\BigOh{\sqrt{n}\log^2{n}}$ time.
