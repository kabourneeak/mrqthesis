\chapter{Preliminaries}
\label{:prelim}

In this chapter, we outline the data structures and other techniques which we rely on for the formulation of solutions to our own contributions.

%------------------------------------------------------------------------------
\section{Range Trees}
\label{:prelim:range-trees}

XXX TODO

\begin{enumerate}
 \item 1D Range trees; say something about how we can store summary information at the inner nodes, like counts, etc.
 \item Multi-level trees
 \item Theorems about preprocessing, size, and query times.  For higher dimensional trees, fractional cascading can also be used.
\end{enumerate}

We use the following theorems on range trees, restated from \cite{debergch5}.

\begin{theorem}
\label{th:rangetree}
Let $P$ be a set of $n$ points in $d$-dimensional space, with $d \geq 2$. A layered range tree for $P$ uses $\BigOh{n \log^{d-1}{n}}$ storage and it can be constructed in $\BigOh{n \log^{d-1}{n}}$ time. With this range tree one can report the points in $P$ that lie in a rectangular query range in $\BigOh{\log^{d-1}{n} + k}$ time where $k$ is the number of reported points.
\end{theorem}

We can augment each node of such a range tree with information about the size of its subtree, requiring only $\BigOh{1}$ additional space and pre-processing per node.  This allows us to count the number of items within a query rectangle rather than reporting them, as stated by the following Corollary.

\begin{corollary}
\label{cor:rangetree}
Let $P$ be a set of $n$ points in $d$-dimensional space, with $d \geq 2$. A layered range tree for $P$ uses $\BigOh{n \log^{d-1}{n}}$ storage and it can be constructed in $\BigOh{n \log^{d-1}{n}}$ time. With this range tree one can count the points in $P$ that lie in a rectangular query range in $\BigOh{\log^{d-1}{n}}$.
\end{corollary}


%------------------------------------------------------------------------------
\section{Chan's Disjoint Set Data-structure}
\label{:prelim:chan}

XXX TODO

\begin{theorem}[Corollary 7.3, part $i$, in Chan\cite{chan2012}]
\label{th:chan}
We can form $O(n)$ canonical subsets of total size $O(n \log{n})$ in $O(n \log{n})$ time, such that the subset of all points inside any query simplex can be reported as a union of disjoint canonical subsets $C_i$ with $\sum_i{\sqrt{|C_i|}} \leq O(\sqrt{n}\log{n})$ in time $O(\sqrt{n}\log{n})$ w.h.p.$(n)$.
\end{theorem}


%------------------------------------------------------------------------------
\section{Fractional Cascading}
\label{:prelim:fractional-cascading}

XXX TODO

I'm not sure that we want to say anything here, actually.


%------------------------------------------------------------------------------
\section{Notation}
\label{:prelim:notation}

Do we have any global notation?  I don't think so;  we don't really want the reader to have to flip back to chapter 2 always anyway.


%------------------------------------------------------------------------------
\section{A word about ``Majority''}

This work was originally inspired by a problem that involved selecting objects which were truly \emph{mostly} inside a query region; that is, where more than half of the object needed to appear in the query region to be counted.  In working on the problems presented in this thesis, however, we have found that many of our techniques work for any fixed proportion of the object, even proportions less than half.

