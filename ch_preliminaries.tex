\chapter{Preliminaries}
\label{:prelim}

In this chapter, we outline the data structures and other techniques which we rely on for the formulation of solutions to our own contributions.

%------------------------------------------------------------------------------
\section{Range Trees}
\label{:prelim:range-trees}

XXX TODO

\begin{enumerate}
 \item 1D Range trees; say something about how we can store summary information at the inner nodes, like counts, etc.
 \item Multi-level trees
 \item Theorems about preprocessing, size, and query times.  For higher dimensional trees, fractional cascading can also be used.
\end{enumerate}

We use the following theorems on range trees, restated from \cite{debergch5}.

\begin{theorem}
\label{th:rangetree}
Let $P$ be a set of $n$ points in $d$-dimensional space, with $d \geq 2$. A layered range tree for $P$ uses $\BigOh{n \log^{d-1}{n}}$ storage and it can be constructed in $\BigOh{n \log^{d-1}{n}}$ time. With this range tree one can report the points in $P$ that lie in a rectangular query range in $\BigOh{\log^{d-1}{n} + k}$ time where $k$ is the number of reported points.
\end{theorem}

We can augment each node of such a range tree with information about the size of its subtree, requiring only $\BigOh{1}$ additional space and pre-processing per node.  This allows us to count the number of items within a query rectangle rather than reporting them, as stated by the following Corollary.

\begin{corollary}
\label{cor:rangetree}
Let $P$ be a set of $n$ points in $d$-dimensional space, with $d \geq 2$. A layered range tree for $P$ uses $\BigOh{n \log^{d-1}{n}}$ storage and it can be constructed in $\BigOh{n \log^{d-1}{n}}$ time. With this range tree one can count the points in $P$ that lie in a rectangular query range in $\BigOh{\log^{d-1}{n}}$.
\end{corollary}


%------------------------------------------------------------------------------
\section{Chan's Disjoint Set Data-structure}
\label{:prelim:chan}

XXX TODO

Restated with $d=2$.

\begin{theorem}[Corollary 7.3, part $i$, in Chan\cite{chan2012}]
\label{th:chan}
We can form $O(n)$ canonical subsets of total size $O(n \log{n})$ in $O(n \log{n})$ time, such that the subset of all points inside any query simplex can be reported as a union of disjoint canonical subsets $C_i$ with $\sum_i{\sqrt{|C_i|}} \leq O(\sqrt{n}\log{n})$ in time $O(\sqrt{n}\log{n})$ w.h.p. $(n)$.
\end{theorem}

XXX TODO Can I just show some example analysis here?

This structure is particularly well-suited to multi-level query structures, as we can associate additional data structures with each of the canonical subsets.

Given an associate data structure which requires $f(n)$ preprocessing space, $g(n)$ preprocessing time, and $h(n)$ query time, we have the following corollary.

\begin{corollary}
\label{cor:chan}
With each of the $k = \BigOh{n}$ canonical subsets $C_1, C_2, \ldots, C_k$ created for a canonical subsets structure, we can associate a data structure with the elements of each subset. If this associated structure requires $\BigOh{n\log^f{n}}$ preprocessing space, $\BigOh{n \log^g{n}}$ preprocessing time, and $\BigOh{\sqrt{n}\log^h{n}}$ query time, where $f, g, h \in O(1), f \leq g$, then the resulting multi-level data structure requires $\BigOh{n\log^{f+1}{n}}$ preprocessing space, $\BigOh{n\log^{g+1}{n}}$ preprocessing time, and $\BigOh{\sqrt{n}\log^{h+1}{n}}$ query time.
\end{corollary}

\begin{proof}
The preprocessing space of the structure requires $\BigOh{n \log{n}}$ space for the canonical subsets structure itself, plus
\[
\begin{split}
\sum_{i=1}^k{|C_i| \log^f{|C_i|}}
&\leq \sum_{i=1}^k{|C_i| \log^f{n}} \\
%
&\leq \BigOh{\log^f{n}} \cdot \sum_{i=1}^k{|C_i|} \\
%
&\leq \BigOh{n\log^{f+1}{n}}
\end{split}
\]

\noindent for the associate structures, for a total space complexity of $\BigOh{n \log{n}} + \BigOh{n\log^{f+1}{n}} = \BigOh{n\log^{f+1}{n}}$.

Preprocessing time is calculated using similar reasoning, and includes the $\BigOh{n \log{n}}$ time for creating the canonical subsets structure itself, plus
\[
\begin{split}
\sum_{i=1}^k{|C_i| \log^g{|C_i|}}
&\leq \sum_{i=1}^k{|C_i| \log^g{n}} \\
%
&\leq \BigOh{\log^g{n}} \cdot \sum_{i=1}^k{|C_i|} \\
%
&\leq \BigOh{n\log^{g+1}{n}}
\end{split}
\]

\noindent for building the associate structures, for a total time complexity of $\BigOh{n \log{n}} + \BigOh{n\log^{g+1}{n}} = \BigOh{n\log^{g+1}{n}}$.

Querying requires $\BigOh{\sqrt{n}\log{n}}$ time to find the $k'$ disjoint canonical subsets representing the elements found by the top-level canonical subsets query, plus
\[
\begin{split}
\sum_{i=1}^{k'}{\BigOh{\sqrt{|C_i|}\log^h{|C_i|}}} 
&\leq \sum_{i=1}^{k'}{\BigOh{\sqrt{|C_i|}\log^h{n}}} \\
%
&\leq \BigOh{\log^h{n}} \cdot \sum_{i=1}^{k'}{\BigOh{\sqrt{|C_i|}}} \\
%
&\leq \BigOh{\sqrt{n}\log^{h+1}{n}} \\
\end{split}
\]

\noindent to query the appropriate associated data structures, for a total query time of 
\[
\BigOh{\sqrt{n}\log{n}} + \BigOh{\sqrt{n}\log^{h+1}{n}} = \BigOh{\sqrt{n}\log^{h+1}{n}}
\]

\end{proof}