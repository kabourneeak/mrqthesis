\chapter{Partial Enclosure Range Searching with Arbitrarily-Oriented Slabs}
\label{:slabs}

In this chapter, we show how we can answer partial enclosure range searches over horizontal line segments with queries in the form of an arbitrarily-oriented slab.
In Section~\ref{:slabs:two}, we extend our solution to a query region which is the intersection of two slabs; this type of query is a generalization of querying with an arbitrarily-oriented rectangle.

%------------------------------------------------------------------------------
\section{Querying with One Slab}
\label{:slabs:one}

In this section, we develop a method for identifying horizontal line segments which are sufficiently enclosed within a single, arbitrarily-oriented query slab.


%------------------------------------------------------------------------------
\subsection{Problem Definition}
\label{:slabs:one:problem-definition}

The formal statement of our problem is as follows.

\begin{problem}
Given a set of $n$ horizontal line segments in the plane, and a fixed parameter $\rho$ such that $0 < \rho \leq 1$, we want to identify those segments which are sufficiently enclosed by an arbitrarily-oriented (i.e., slanted) slab $Q$ to satisfy the partial enclosure property.
A segment $s$ satisfies this property if and only if $|s \cap Q| \geq \rho \cdot |s|$.
\end{problem}

Throughout this section, we use the following definitions. 
Let $S$ be a set of $n$ horizontal line segments.
Each segment $s_i \in S, 1 \leq i \leq n$ is defined by three values $a_i, b_i, l_i$ which in turn define the endpoints of the segment as $p_i = (a_i, b_i)$ and $q_i = (a_i + l_i, b_i)$. 
When discussing a single, general segment, we omit the $i$ subscripts for clarity.

A query $Q$ is given as three inputs: $\alpha, \beta, w$ which we use to define an arbitrarily-oriented slab.
The left edge of the slab is defined by $L_1 : y = \alpha x + \beta$, while the right edge defined by $L_2: y = \alpha x + \beta - \alpha w$. 
Thus, $w$ is the horizontal width of $Q$.

We say that $s_i \in_\rho Q$ if and only if $s_i$ satisfies the partial enclosure property with respect to $Q$, otherwise $s \not \in_\rho Q$.


%------------------------------------------------------------------------------
\subsection{Identifying the Segments}
\label{:slabs:one:approach}

Identifying segments which satisfy the partial enclosure property requires three broad steps:

\begin{enumerate}
 \item Restrict segments to those which are ``not too long'' to fit sufficiently inside $Q$.

 \item Classify all segments depending on the location of their left endpoints. We are interested in those segments whose left endpoints are left of $L_1$ and those which are between $L_1$ and $L_2$.

 \item For each class of interesting segments, test an appropriate partial enclosure expression.

\end{enumerate}

To answer these queries, we will use a multi-level canonical sets data structure, as described in Section~\ref{:prelim:chan}. 
We describe the different steps of the query in more detail next, while Section~\ref{:slabs:one:analysis} describes the construction and analysis of the data structure.


%------------------------------------------------------------------------------
\subsubsection{Restrict Length}
\label{:slabs:one:details:restrict}

The first step of the query is to perform a length test, as it simplifies some later queries.
We want to identify all segments which are not so long that they could never be sufficiently enclosed by $Q$, regardless of their position.
With the query parameter $w$ given, we know that only segments with length $l \leq \frac{w}{\rho}$ can ever be selected.
Reworking the equation to $w \geq \rho l$ provides a 1-dimensional orthogonal range to query.
Let $S_1 = \{ s \in S \st \rho l \leq w \}$.


%------------------------------------------------------------------------------
\subsubsection{Classify Endpoints}
\label{:slabs:one:details:classify}

There are three different regions where we may find the left endpoints of our line segments: (1) left of $L_1$, (2) between $L_1$ and $L_2$, and (3) right of $L_2$.
However, only those segments belonging to cases (1) and (2) are interesting to us, since any segment in case (3) cannot intersect $Q$ at all. 
We will see that partitioning segments as left or right of $L_1$ is sufficient, as we can discriminate between cases (2) and (3) when testing partial enclosure expressions.

Identifying segments whose left endpoints appear to one side of $L_1$ can be accomplished using a half-plane query directly on the positions of the left endpoints and the definition of $L_1$ itself.
Let $S_L = \{ s \in S_1 \st p \text{ is left of } L_1 \}$, and let $S_R = \{ s \in S_1 \st p \text{ is right of } L_1 \}$.


%------------------------------------------------------------------------------
\subsubsection{Check the Partial Enclosure Property}
\label{:slabs:one:details:pep}

For each of $S_L$ and $S_R$, the final query step is to identify those segments which satisfy the partial enclosure property.

\begin{figure}[t]
\begin{center}
  \includegraphics[width=0.80\textwidth]{figures/fig_aoq_left_l1}
  \caption{Segments whose left endpoints are left of $L_1$.}
  \label{fig:slabs:one:aoq_left_l1}
\end{center}
\end{figure}

\paragraph{Set $S_L$, left endpoint left of $L_1$.} There are 4 cases of segments whose left endpoints are left of $L_1$, as shown in Figure \ref{fig:slabs:one:aoq_left_l1}.

\begin{enumerate}
 \item The entire segment is left of $L_1$ (and should not be counted),

 \item The segment intersects $L_1$, and the segment is sufficiently enclosed by $Q$,

 \item The segment intersects $L_1$, but the segment is insufficiently enclosed by $Q$ (and should not be counted),

 \item The segment intersects both $L_1$ and $L_2$. 

\end{enumerate}

Given a segment $s \in S_L$ with left endpoint $p = (a,b)$, let $\iline{s}$ be the line through $s$, given by the equation $y = b$ (the slope is 0). 
Let $(a', b)$ be the intersection point of $\iline{s}$ and $L_1$. 
Solving the equations of the two lines gives $a' = \frac{b - \beta}{\alpha}$. 
It is sufficient to show that $s \in_\rho Q$ when ``not too much of $s$ is outside of $Q$'', which we can test with the expression $a' - a < (1 - \rho)l$.

\begin{proof}
We look at each case to show that this single expression is enough to identify all segments correctly. 
First, the test directly identifies segments belonging to cases (2) or (3) since $a' - a$ is precisely the amount of $s$ outside of $Q$. 
This test will also reject segments, as we desire, from case (1) since $a' - a > l$ and cannot satisfy the partial enclosure property for any allowed value of $\rho$. 

Finally, this test is also sufficient for case (4) owing to the length restriction step we perform in Section~\ref{:slabs:one:details:restrict}. 
If a segment is not too far left of $L_1$, then either it crosses only $L_1$, and case (2) holds, or it crosses $L_1$ and $L_2$. 
In the latter case we know that $|s| < \frac{w}{\rho}$, where $w$ is the width of the query slab. 
Since any segment in case (4) has the property $|s \cap Q| = w$, this implies that $s \in_\rho Q$.
\end{proof}

We test for segments satisfying $a' - a < (1 - \rho)l$ by transforming it to a half-plane query. Expanding this expression gives:
\[
\begin{split}
a' - a &< (1 - \rho)l \\
%
\frac{b - \beta}{\alpha} - a &< (1 - \rho)l \\
%
a + (1 - \rho)l &> \frac{1}{\alpha} b - \frac{\beta}{\alpha}  \\
\end{split}
\]

From this inequality, we can construct an appropriate dual-space to perform the half-plane query. We map each segment $s$ to a dual-point with coordinates $(b, a + (1-\rho)l)$. 
Segments matching the partial enclosure expression then correspond to the dual-points satisfying the half-plane $y > \frac{1}{\alpha} - \frac{\beta}{\alpha}$.

%------------------------------------------------------------------------------

\paragraph{Set $S_R$, left endpoint right of $L_1$.}
There are 4 cases of segments whose left-endpoints are right of $L_1$, as shown in Figure \ref{fig:slabs:one:aoq_left_l2}.

\begin{figure}[t]
\begin{center}
  \includegraphics[width=0.80\textwidth]{figures/fig_aoq_left_l2}
  \caption{Segments whose left endpoints are between $L_1$ and $L_2$.}
  \label{fig:slabs:one:aoq_left_l2}
\end{center}
\end{figure}


\begin{enumerate}
 \item The entire segment is between $L_1$ and $L_2$. Since $s \in Q \imply s \in_\rho Q$, this segment should be counted.

 \item The entire segment is right of $L_2$. No part of $s$ is in $Q$, so it should not be counted.

 \item The segment intersects $L_2$, and $s \in_\rho Q$.

 \item The segment intersects $L_2$, but $s \not \in_\rho Q$.

\end{enumerate}

For each $s \in S_R$, we test the partial enclosure property by checking that ``enough of the segment is inside of $Q$''.
Just as we did with the segments of $S_L$, let $\iline{s}$ be the horizontal line through $s$.
Let $(a'', b)$ be the intersection point of $\iline{s}$ with $L_2$.
Solving their respective equations gives $a'' = \frac{b - \beta}{\alpha} + w$.  
We will show that a segment satisfies the partial enclosure property when $a'' - a \geq \rho l$.

\begin{proof}
If $s \in S_R$ belongs to case (1), we have that $a'' - a > l$, which is certainly greater than $\rho l$. 
If $s$ belongs to case (2), then $a'' - a < 0$ since $a$ is right of $a''$, and $s$ will not be counted.
Finally, the correctness of this test is straight-forward when $s$ is in cases (3) or (4), since the expression directly measures $|s \cap Q|$.
\end{proof}

We test for segments satisfying $a'' - a \geq \rho l$ using a half-plane query.  Expanding this expression gives:
\[
\begin{split}
a'' - a &\geq \rho l \\
%
\frac{b - \beta}{\alpha} + w - a &\geq \rho l \\
%
\rho l + a &\leq \frac{1}{\alpha} b - \frac{\beta}{\alpha} + w
%
\end{split}
\]

From this inequality, we can construct an appropriate dual-space to perform the half-plane query.  
We map each segment $s$ to a dual-point with coordinates $(b, \rho l + a)$. 
Segments matching the partial enclosure expression then correspond to the dual-points satisfying the half-plane $y \leq \frac{1}{\alpha} x - \frac{\beta}{\alpha} + w$.

%------------------------------------------------------------------------------
\subsection{Construction and Analysis}
\label{:slabs:one:analysis}

We will use the multi-level canonical sets data structure described in Section~\ref{:prelim:chan} to perform parts of this query.
Each of the three steps given in Section~\ref{:slabs:one:approach} will correspond to one nested level of the final data structure.
It is easiest to describe the structure inside-out, so we begin with the innermost structure.


%------------------------------------------------------------------------------
\subsubsection{Length Restriction}

The innermost structure answers the length restriction step of the overall query. This is easily answered using a 1-dimensional range tree, as described in Section~\ref{:prelim:range-trees}, keyed on the segment lengths. The following lemma summarizes the time and space requirements of this structure.

\begin{lemma}
\label{lem:slabs:one:step1}
Given a set of $n$ horizontal line segments, we can count all segments with length at most $\frac{w}{\rho}$ in $\BigOh{\log{n}}$ time, using a data structure of size $\BigOh{n}$, which can be built in $\BigOh{n \log{n}}$ preprocessing time.
\end{lemma}


% -----------------------------------------------------------------------------
\subsubsection{Partial Enclosure Property}

To identify segments satisfying the partial enclosure property, we use the half-plane expressions we developed in Section \ref{:slabs:one:details:pep}.
There are two expressions we need to test, depending on whether the left endpoint of a segment is left or right of $L_1$.
By Theorem~\ref{:prelim:chan}, we can answer this type of half-plane query directly using a canonical subsets data structure, yielding the following lemma applicable to both cases.

\begin{lemma}
\label{lem:slabs:one:step2a}
Given a set of $n$ horizontal line segments, we can locate all segments which are not ``too much'' to one side of a query line in $\BigOh{\sqrt{n}\log{n}}$ time, using a data structure of size $\BigOh{n\log{n}}$, which can be built in $\BigOh{n\log{n}}$ time.
\end{lemma}

With each subset created for this structure, we will associate the structure required for Lemma~\ref{lem:slabs:one:step1}.
Applying Corollary~\ref{cor:multichan} gives us the following result.

\begin{lemma}
\label{lem:slabs:one:step2b}
Given a set of $n$ horizontal line segments, we can locate all segments which are not ``too much'' to one side of a query line \emph{and} do not exceed a maximum length in $\BigOh{\sqrt{n}\log^2{n}}$ time, using a data structure of size $\BigOh{n\log{n}}$, which can be built in $\BigOh{n\log^2{n}}$ time.
\end{lemma}


% -----------------------------------------------------------------------------
\subsubsection{Left-endpoint classification}

We need to classify the left-endpoints of the segments as left or right of $L_1$. 
By Theorem~\ref{th:chan}, each of these queries can be answered by a canonical subsets data structure.
With each subset created by this structure, we will associate the structure from Lemma~\ref{lem:slabs:one:step2b}. 
Applying Corollary~\ref{cor:multichan} gives us the following result.

\begin{lemma}
\label{lem:slabs:one:step3}
Given a set of $n$ horizontal line segments, we can locate all segments whose left endpoints are to one side of a line, which are not ``too much'' to one side of a query line, \emph{and} do not exceed a maximum length in $\BigOh{\sqrt{n}\log^3{n}}$ time, using a data structure of size $\BigOh{n\log{n}^2}$, which can be built in $\BigOh{n\log^3{n}}$ time.
\end{lemma}

% -----------------------------------------------------------------------------
\subsubsection{Combining the Steps}

To fully answer any query $Q$, we need to preprocess two such data structures, choosing our expressions appropriately for the left and right cases.
During query time, we will query both structures and combine their results.
The following theorem summarizes the overall solution.

\begin{theorem}
\label{th:slabs:one}
Given a set of $n$ horizontal line segments, we can identify all segments which satisfy the partial enclosure property for an arbitrarily-oriented query slab in $\BigOh{\sqrt{n}\log^3{n}}$ time, using a data structure of size $\BigOh{n\log^2{n}}$, requiring $\BigOh{n\log^3{n}}$ preprocessing time.
\end{theorem}


%------------------------------------------------------------------------------
%------------------------------------------------------------------------------
\section{Querying with Two Slabs}
\label{:slabs:two}

In this section, we consider a variation of the slab query. 
Our query is input as two slabs, and we are interested in those segments which satisfy the partial enclosure property with respect to their intersection.
When the two slabs are orthogonal to each other, this method will solve the problem of finding segments sufficiently enclosed by an arbitrarily-oriented rectangle.  
The overall approach is very similar to Section~\ref{:slabs:one}, and the data structure we will build to answer these queries is again based on the canonical sets structure outlined in Section~\ref{:prelim:chan}.


% -----------------------------------------------------------------------------
\subsection{Problem Definition}
\label{:slabs:two:problem}

The formal statement of this problem is as follows.

\begin{problem}
Given a set of $n$ horizontal line segments in the plane, and a fixed parameter $\rho$ such that $0 < \rho \leq 1$, we want to identify those segments which are sufficiently enclosed by an arbitrarily-oriented parallelogram $Q$ to satisfy the partial enclosure property.
A segment $s$ satisfies this property if and only if $|s \cap Q| \geq \rho \cdot |s|$.
\end{problem}

Throughout this section, we use the following definitions. 
Let $S$ be a set of $n$ horizontal line segments.
Each segment $s_i \in S, 1 \leq i \leq n$ is defined by three values $a_i, b_i, l_i$ which in turn define the endpoints of the segment as $p_i = (a_i, b_i)$ and $q_i = (a_i + l_i, b_i)$.

A query region is an arbitrarily-oriented parallelogram defined by the intersection of two slabs, $S_p$ and $S_n$, which have positive and negative slopes, respectively. See Figure \ref{fig:slabs:two:ds}.

\begin{figure}[t]
\begin{center}
  \includegraphics[width=0.90\textwidth]{figures/fig_ds_slabs}
  \caption{A query parallelogram $Q$ formed by the inputs $\alpha$, $\beta$, $w_p$,
  $\gamma$, $\delta$, and $w_n$.}
  \label{fig:slabs:two:ds}
\end{center}
\end{figure}

Specifically, a query is given as six inputs $\alpha, \beta, w_p, \gamma, \delta, w_n$, where $\alpha > 0$, $w_p > 0$, $w_n > 0$, and $\gamma < 0$. With these inputs, we define:

\begin{itemize}
 \item The line $L_1 : y = \alpha x + \beta$, the left edge of $S_p$,

 \item The line $L_2 : y = \alpha (x - w_p) + \beta$, the right edge of $S_p$,

 \item The line $L_3 : y = \gamma x + \delta$, the left edge of $S_n$,

 \item The line $L_4 : y = \gamma (x + w_n) + \delta$, the right edge of $S_n$,

 \item The slab $S_p = \{ u \in \mathbb{R}^2 | (u \text{ is on or right of } L_1) \wedge (u \text{ is on or left of } L_2) \}$, a slab with positive slope,

 \item The slab $S_n = \{ u \in \mathbb{R}^2 | (u \text{ is on or right of } L_3) \wedge (u \text{ is on or left of } L_4) \}$, a slab with negative slope,

 \item The parallelogram $Q = S_p \cap S_n$.

\end{itemize}

We say that $s_i \in_\rho Q$ if and only if $s_i$ satisfies the partial enclosure property with respect to $Q$, otherwise $s \not \in_\rho Q$.


% -----------------------------------------------------------------------------
\subsection{Identifying the Segments}
\label{:slabs:two:approach}

Just as in the single slab problem, identifying segments satisfying the partial enclosure property is accomplished by classifying the endpoints of our segments by how they interact with $Q$ and then testing an appropriate partial enclosure expression. 
We first describe the different steps of the query in more detail, and then describe the construction and analysis of an appropriate data structure.


% -----------------------------------------------------------------------------
\subsubsection{Decomposition}

\begin{figure}[t]
\begin{center}
  \includegraphics[width=0.85\textwidth]{figures/fig_ds_wpwn}
  \caption[Decomposition of the query region $Q$.]{Decomposition of the query region $Q$. The orientation of $Q$ depends on the relative widths of the slabs which define it.}
  \label{fig:slabs:two:wpwn}
\end{center}
\end{figure}

The query $Q$ is decomposed into three regions, $Q_a$, $Q_b$, and $Q_c$, by extending horizontal lines through each vertex of $Q$.
$Q_a$ and $Q_c$ are triangular regions, while $Q_b$ is a parallelogram. 
$Q_a$ is always defined by $L_1$ on the left and $L_4$ on the right. 
Likewise, $Q_c$, is always defined by $L_3$ on the left and $L_2$ on the right. 
The definition of $Q_b$ depends on the overall orientation of $Q$, which depends on the relationship between $w_p$ and $w_n$.
Specifically, $Q_b$ is defined by $L_1$ and $L_2$ when $w_p < w_n$ and defined by $L_3$ and $L_4$ when $w_p > w_n$.
When $w_p = w_n$, $Q_b$ disappears, and we don't need to continue further query steps on it.


\subsubsection{Endpoint Classification}

Once we know the definitions of each region, we can proceed with endpoint classification.
The goal of this step is to identify which partial enclosure expression is appropriate to test with each line segment by considering which borders of $Q$ it interacts with. 
Each region has a left, center, and right classification zone, $Z_L$, $Z_C$, and $Z_R$, where the center zone is the query region itself.
Figure~\ref{fig:slabs:two:endpoint} gives an illustration; $Z_L$, $Z_C$, and $Z_R$ are coloured blue, green, and red, respectively.

\begin{figure}[t]
\begin{center}
  \includegraphics[width=1.00\textwidth]{figures/fig_ds_endpoint}
  \caption[Classification zones for $Q$.]{Classification zones for each region of $Q$. Each region has a left, center, and right zone, coloured blue, green, and red, respectively, where we may find segments which need further testing.}
  \label{fig:slabs:two:endpoint}
\end{center}
\end{figure}

For $Q_a$, $Z_L$ is aligned with the base of $Q_a$, and extends up $L_1$. 
$Z_R$ is also aligned with the base of $Q_a$, and extends up $L_4$.
Both of these zones are open away from $Q$.
The zones for $Q_c$ are symmetric, being aligned with the top of the region, and with $Z_L$ and $Z_R$ extending down $L_3$ and $L_2$, respectively.

For $Q_b$, the classification zones we use depend on the orientation of the region.
When $w_p < w_n$, $Z_L$ is aligned with the base of $Q_b$ and extends up $L_1$, while $Z_R$ is aligned with the top of $Q_b$ and extends down $L_2$.
Conversely, when $w_p > w_n$, $Z_L$ is aligned with the top of $Q_b$ and extends down $L_3$, while $Z_R$ is aligned with the bottom of $Q_b$ and extends up $L_4$.
The former case is what is illustrated in Figure~\ref{fig:slabs:two:endpoint}.
$Z_C$ is $Q_b$ itself, which we decompose into two triangular subzones so that all of our zones are triangular in nature.
During the query, any time that we consider $Z_C$ for $Q_b$, we will query both subzones and consider their union.

Any segment interacting with a particular query region must have its endpoints in one of 6 combinations of classification zones: $(Z_L, Z_L)$, $(Z_L, Z_C)$, $(Z_L, Z_R)$, $(Z_C, Z_C)$, $(Z_C, Z_R)$, or $(Z_R, Z_R)$. 
For a segment $s \in S$ with endpoints $p$ and $q$:
\begin{itemize}
\item If $p, q \in (Z_C, Z_C)$, then $s$ is entirely inside $Q$.
\item If $p, q \in (Z_L, Z_L)$, or $p, q \in (Z_R, Z_R)$ then $s$ is entirely outside $Q$.
\item Otherwise, $s$ crosses one or both boundaries of the query region and we need to test partial enclosure expressions.
\end{itemize}


\subsubsection{Partial Enclosure Property}

We begin by examining $Q_a$ in detail.
Let $s \in S$ be a horizontal line segment and let $\iline{s}$ be the line through $s$, defined by the equation $y = b$.
Let $(a', b) = \iline{s} \cap L_1$ and $(a'', b) = \iline{s} \cap L_4$, be the intersection points of $\iline{s}$ with $L_1$ and $L_4$, respectively, then $a' = \frac{b - \beta}{\alpha}$ and $a'' = \frac{b - \delta}{\gamma} + w_n$.
We have three combinations of classification zones that need further testing.

For the $(Z_L, Z_C)$ case, we know that $s$ only crosses $L_1$.
We check that not too much of $s$ is outside of $Q_a$, giving the following partial enclosure expression.
\[
\begin{split}
a' - a &< (1 - \rho)l \\
%
\frac{b - \beta}{\alpha} - a &< (1 - \rho)l \\
%
b \frac{1}{\alpha} - \frac{\beta}{\alpha} &< a + (1 - \rho)l \\
%
\end{split}
\]

\noindent We can query for segments matching this expression by performing a half-plane query on an appropriate dual-space, just as we did in Section~\ref{:slabs:one:details:pep}. 
For this expression in particular, we map each segment $s$ to a dual-point with coordinates $(b, a + (1-\rho)l)$, and query with the half-plane $y > \frac{1}{\alpha} x - \frac{\beta}{\alpha}$.

For the $(Z_C, Z_R)$ case, we know that $s$ only crosses $L_4$. 
We check that enough of $s$ is inside $Q_a$, which gives the following partial enclosure expression.
\[
\begin{split}
a'' - a &\geq \rho l \\
%
\frac{b - \delta}{\gamma} + w_n - a &\geq \rho l \\
%
b \cdot \frac{1}{\gamma} - \frac{\delta}{\gamma} + w_n &\geq a + \rho l \\
%
\end{split}
\]

\noindent We can test this expression by mapping each segment $s$ to a dual-point with coordinates $(b, a + \rho l)$ and then querying with the half-plane $y \leq \frac{1}{\gamma} x - \frac{\delta}{\gamma} + w_n$.

Finally, for the $(Z_L, Z_R)$ case, we know that both endpoints of $s$ are outside of $Q_a$, so we only need to measure the width of $s \cap Q_a$.  
Specifically, we require that:
\[
\begin{split} 
a'' - a' &\geq \rho l \\
%
\frac{b - \delta}{\gamma} + w_n - \frac{b - \beta}{\alpha} &\geq \rho l \\
%
\frac{b}{\gamma} - \frac{\delta}{\gamma} + w_n - \frac{b}{\alpha} + \frac{\beta}{\alpha} &\geq \rho l \\
%
b \cdot \left ( \frac{1}{\gamma} - \frac{1}{\alpha} \right ) + \left ( \frac{\beta}{\alpha} - \frac{\delta}{\gamma} + w_n \right ) &\geq \rho l \\
%
\end{split}
\]

\noindent We can test this expression by mapping each segment $s$ to a dual-point with coordinates $(b, \rho l)$ and then querying with the following half-plane.
\[
y \leq \left ( \frac{1}{\gamma} - \frac{1}{\alpha} \right ) \cdot x + \left ( \frac{\beta}{\alpha} - \frac{\delta}{\gamma} + w_n \right )
\]

\noindent Classification into the left and right zones is somewhat ``rough'', as the zone continues above $Q_a$ itself. 
This is not a problem in the $(Z_L, Z_C)$ and $(Z_C, Z_R)$ cases since one endpoint of $s$ is classified directly in the closed zone $Z_C$ and the segments are horizontal.
However, it can happen that a segment classified into $(Z_L, Z_R)$ is entirely above $Q_a$. 
In this case, since we are measuring $a'' - a'$, and $a'' < a'$ above the apex of $Q_a$, the expression will be negative and the segment will be rejected.

The partial enclosure expressions for $Q_b$ and $Q_c$ are developed using exactly the same reasoning as for $Q_a$, differing only by which of the lines $L_1$, $L_2$, $L_3$, and $L_4$ we use to define $a'$ and $a''$.


% -----------------------------------------------------------------------------
\subsection{Construction and Analysis}
\label{:slabs:two:analysis}

We will use a multi-level query structure for this problem, just as we did for the single slab query.
Each of the steps given in Section~\ref{:slabs:two:approach} will correspond to one nested level of the data structure.
It is easiest to describe the structure inside-out, so we begin with the innermost structure.

\subsubsection{Check Partial Enclosure Expression}

To identify segments satisfying a partial enclosure expression, we use the half-plane dual-spaces we developed in Section~\ref{:slabs:two:approach}. 
By Theorem~\ref{:prelim:chan}, we can answer such a half-plane query directly using a canonical subsets data structure, yielding the following lemma.

\begin{lemma}
\label{lem:slabs:two:step1}
Given a set of $n$ horizontal line segments, we can identify all segments which are not ``too much'' to one side of a query line in $\BigOh{\sqrt{n}\log{n}}$ time, using a data structure of size $\BigOh{n\log{n}}$, which can be built in $\BigOh{n\log{n}}$ time.
\end{lemma}

\subsubsection{Classify Right Endpoint}

We need to be able to classify the right endpoints of our segments into any of the classification zones of each query region.
Each of these regions are triangular in nature and can be queried with a canonical subsets data structure. 
With each subset created by this structure, we will associate the structure from Lemma~\ref{lem:slabs:two:step1}. 
Applying Corollary~\ref{cor:multichan} gives us the following result.

\begin{lemma}
\label{lem:slabs:two:step2}
Given a set of $n$ horizontal line segments, we can identify all segments which have their right endpoint in a query triangle, and which are not ``too much'' to one side of a query line, in $\BigOh{\sqrt{n}\log^2{n}}$ time, using a data structure of size $\BigOh{n\log^2{n}}$, which can be built in $\BigOh{n\log^2{n}}$ time.
\end{lemma}


\subsubsection{Classify Left Endpoint}

We need to be able to classify the left endpoints of our segments into any of the classification zones as well, which will use another canonical subsets data structure.
With each subset of this structure, we continue to build up our multi-level structure by associating the structure from Lemma~\ref{lem:slabs:two:step2}. 
By applying Corollary~\ref{cor:multichan} again, we have the following lemma.

\begin{lemma}
\label{lem:slabs:two:step3}
Given a set of $n$ horizontal line segments, we can identify all segments which have their left and right endpoints in given query triangles, and which are not ``too much'' to one side of a query line, in $\BigOh{\sqrt{n}\log^3{n}}$ time, using a data structure of size $\BigOh{n\log^3{n}}$, which can be built in $\BigOh{n\log^3{n}}$ time.
\end{lemma}


\subsubsection{Combining the Steps}

To fully answer a query $Q$, we only need one instance of the structures for classifying the left and right endpoints, since these structures can support all of the different triangular classification queries we need to perform.
We will need to preprocess several versions of the innermost level which determines the partial enclosure expression, however.

During query time, we check $(Z_C, Z_C)$, $(Z_L, Z_C)$, $(Z_C, Z_R)$ and $(Z_L, Z_R)$ for every query region. 
For any of the latter three zone combinations which are non-empty, we check the innermost structure corresponding to the appropriate partial enclosure expression.
The following theorem summarizes this process.

\begin{theorem}
\label{th:slabs:two}
Given a set of $n$ horizontal line segments, we can identify all segments which satisfy the partial enclosure property with respect to the intersection of two query slabs in $\BigOh{\sqrt{n}\log^3{n}}$ time, using a data structure of size $\BigOh{n\log^3{n}}$, requiring $\BigOh{n\log^3{n}}$ preprocessing time.
\end{theorem}


%------------------------------------------------------------------------------
%------------------------------------------------------------------------------
\section{Notes about Arbitrarily-Oriented Segments}
\label{:slabs:remarks}

Our general approach can be extended to identify arbitrarily-oriented line segments which satisfy the partial enclosure property for one or two slabs, however, the resulting structure is very ``case heavy'' and involves a large number of query variables.

Arbitrarily-oriented segments can cross through the borders of $Q$ in many more ways than horizontal ones.
This prevents us from partitioning the environment into a simple set of classification zones, and increases the number of cases to consider.

The partial enclosure expressions for each case will also involve higher degree polynomials.
For example, a segment $s$ which crosses $L_1$ and $L_2$ has intersection points $p' = (a', b')$ with $L_1$ and $q' = (a'', b'')$ with $L_2$.  
To consider how much of the segment is inside the slab, we need to calculate the length of the segment from $p'$ to $q'$, as follows.
\[
\begin{split}
\|p'q'\|^2 
&= \left( \sqrt{(a'' - a')^2 + (b'' - b')^2} \right)^2 \\
%
&= (a'' - a')^2 + (b'' - b')^2 \\
%
&= (a'')^2 - 2a'a'' + (a')^2 + (b'')^2 - 2b'b'' + (b')^2
\end{split}
\]

Each term in the above expression represents several query variables when we consider the actual values for $a'$, $b'$, $a''$, and $b''$.  
Let $\iline{s}$ be the line through $s$, and assume that it is defined as $y = mx + t$, then
\begin{align*}
a'  &= \frac{\beta - t}{m - \alpha}
&b'  &= m \frac{\beta - t}{m - \alpha} + t \\
a'' &= \frac{\beta - t - \alpha w}{m - \alpha} 
&b'' &= m \frac{\beta - t - \alpha w}{m - \alpha} + t \\
\end{align*}

Multiplying and squaring these expressions to calculate $\|p'q'\|^2$ results in a high degree polynomial with respect to the number of independent query variable expressions.
The situation worsens when we consider intersections between $L_1$ or $L_2$ with $L_3$ or $L_4$, as we have to consider $\gamma$ and $\delta$ as well.
Such an expression \emph{can} be queried with a canonical subset structure, but the space and query times suffer.

%------------------------------------------------------------------------------
%------------------------------------------------------------------------------
\section{Conclusion}
\label{:slabs:concl}

In this chapter, we have developed methods for identifying horizontal segments which satisfy the partial enclosure property with respect to an arbitrary slab or parallelogram.
The latter case is a generalization of arbitrarily-oriented rectangles when the two slabs defining the parallelogram are orthogonal to each other.

In future work, we would like to find a more straight-forward method for relaxing the horizontal restriction on the input segments.  
A less case-heavy method for querying arbitrarily-oriented segments against slabs would automatically improve our results from Section~\ref{:rectangles:ao}, as well.
