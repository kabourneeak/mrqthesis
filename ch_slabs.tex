\chapter{Majority Range Queries with Arbitrarily-Oriented Slabs}
\label{:slabs}

In this chapter, we show how we can answer majority range queries over horizontal line segments for queries in the form of an arbitrarily-oriented slab in the plane.  In Section~\ref{:slabs:two}, we extend our solution to allow a query region which is the intersection of two slabs; this type of query is a generalization of querying with an arbitrarily-oriented rectangle.

%------------------------------------------------------------------------------
\section{Querying with One Slab}
\label{:slabs:one}

In this section, we develop a method for identifying horizontal line segments which are sufficiently enclosed within a single, arbitrarily-oriented query slab.


%------------------------------------------------------------------------------
\subsection{Problem Definition}
\label{:slabs:one:problem-definition}

The formal statement of our problem is as follows.

\begin{problem}
We are given a set $S$ of $n$ horizontal line segments in the plane, and a fixed parameter $\rho$ such that $0 < \rho \leq 1$. Our query $Q$ is an arbitrarily-oriented (i.e., slanted) slab. We want to count the number of segments $s \in S$ satisfying the property $|s \cap Q| \geq \rho \cdot |s|$.
\end{problem}

Throughout this section, we use the following definitions. Each segment $s_i \in S, 1 \leq i \leq n$ is defined by three values $a_i, b_i, l_i$ which in turn define the endpoints of the segment as $p_i = (a_i, b_i)$ and $q_i = (a_i + l_i, b_i)$. When discussing a single, general segment, we drop the $i$ subscripts for clarity.

A query $Q$ is given as three inputs: $\alpha, \beta, w$ which we use to define an arbitrarily-oriented (i.e., slanted) slab. The left edge of the slab is defined by $L_1 : y = \alpha x + \beta$, while the right edge defined by $L_2: y = \alpha x + \beta - \alpha w$. Thus, $w$ is the horizontal width of $Q$.

A segment $s \in S$ with length $|s|$ is sufficiently enclosed in $Q$ if and only if at least $\rho \cdot |s|$ of the segment is within $Q$. More precisely, we say that $s \in_\rho Q \Leftrightarrow |s \cap Q| \geq \rho \cdot |s|$. Otherwise, $s \not \in_\rho Q$.


%------------------------------------------------------------------------------
\subsection{Identifying the Segments}
\label{:slabs:one:approach}

Identifying the segments which meet our enclosure property takes place in three broad steps:

\begin{enumerate}
 \item Restrict ourselves to segments which are ``not too long'' to fit mostly inside $Q$.

 \item Classify all segments depending on the location of their left endpoints. We are interested in those segments whose endpoints are left of $L_1$ and those which are between $L_1$ and $L_2$.

 \item For each class of interesting segments, we test their majority property.

\end{enumerate}

To answer these queries, we will use a multi-level \emph{canonical sets} data structure, as described in Section~\ref{:prelim:chan}. We describe the different steps of the query in more detail here while Section~\ref{:slabs:one:analysis} describes the construction and analysis of the query data structure.


%------------------------------------------------------------------------------
\subsubsection{Restrict Length}
\label{:slabs:one:details:restrict}

The first step of the query is to perform a length test, as it simplifies some later queries.  We want to identify all segments which are not so long that they could never be sufficiently enclosed by $Q$, regardless of their position. With the query parameter $w$ given, we know that only segments with length $l \leq \frac{w}{\rho}$ can ever be selected. Reworking the equation to $w \geq \rho l$ provides us a 1-dimensional orthogonal range to query. Let $S_1 = \{ s \in S \st l \rho \leq w \}$.


%------------------------------------------------------------------------------
\subsubsection{Classify Endpoints}
\label{:slabs:one:details:classify}

Broadly speaking, there are three different regions where we may find the left endpoints of our line segments:
\begin{enumerate}
 \item Left of $L_1$,
 \item Between $L_1$ and $L_2$, and
 \item Right of $L_2$.
\end{enumerate}

However, only those segments belonging to cases 1 and 2 are interesting to us, since any segments whose left endpoint appears right of $L_2$ cannot be sufficiently enclosed by $Q$. We will see that simply partitioning segments as left or right of $L_1$ is sufficient, as we can discriminate between cases 2 and 3 when testing the majority property.

Identifying segments whose left endpoints appear to one side of $L_1$ can be accomplished using a half-plane query directly on the positions of the left endpoints and with $L_1$ itself.  Let $S_L = \{ s \in S_1 \st p \text{ is left of } L_1 \}$, and let $S_R = \{ s \in S_1 \st p \text{ is right of } L_1 \}$.


%------------------------------------------------------------------------------
\subsubsection{Check Majority Property}
\label{:slabs:one:details:majority}

For each set $S_L$ and $S_R$, the final query step is to identify those segments which satisfy the majority property.

\begin{figure}[t]
\begin{center}
  \includegraphics[width=0.80\textwidth]{figures/fig_aoq_left_l1}
  \caption{Segments whose left endpoints are left of $L_1$.}
  \label{fig:slabs:one:aoq_left_l1}
\end{center}
\end{figure}

\paragraph{Set $S_L$, left endpoint left of $L_1$.} There are 4 subcases of segments whose left endpoints are left of $L_1$, as shown in Figure \ref{fig:slabs:one:aoq_left_l1}.

\begin{enumerate}
 \item The entire segment is left of $L_1$ (and should not be counted),

 \item The segment intersects $L_1$, and the segment is mostly enclosed by $Q$,

 \item The segment intersects $L_1$, but the segment is not mostly enclosed by $Q$ (and should not be counted),

 \item The segment intersects both $L_1$ and $L_2$. 

\end{enumerate}

For these segments, we test their majority property as follows. Given a segment $s$ with left endpoint $p = (a,b)$, let $\iline{s}$ be the line through $s$, given by the equation $y = b$ (the slope is 0). Let $(a', b)$ be the intersection point of $\iline{s}$ and $L_1$. Solving their respective equations gives $a' = \frac{b - \beta}{\alpha}$. It is sufficient to show that $s \in_\rho Q$ when ``not too much of $s$ is outside of $Q$'', that is, that $a' - a < (1 - \rho)l$. 

\begin{proof}
We look at each subcase to show that this single test is enough to identify all segments correctly. First, the test directly identifies segments belonging to subcases 2 or 3. This test will also reject segments, as we desire, from subcase 1 since $a' - a > l$ and cannot satisfy the majority property for any allowed value of $\rho$. 

Finally, this test is also sufficient for the 4th subcase owing to the length restriction step we perform in Section\ref{:slabs:one:details:restrict}. That is, after the length restriction step, we know that $|s| < \frac{w}{\rho}$, where $w$ is the width of the query slab. Since any segment in subcase 4 has the property $|s \cap Q| = w$, this implies that $s \in_\rho Q$.
\end{proof}

We test for segments satisfying $a' - a < (1 - \rho)l$ using a half-plane query. Expanding this expression gives:
\[
\begin{split}
a' - a &< (1 - \rho)l \\
%
\frac{b - \beta}{\alpha} - a &< (1 - \rho)l \\
%
a + (1 - \rho)l &> \frac{1}{\alpha} b - \frac{\beta}{\alpha}  \\
\end{split}
\]

From this inequality, we can construct an appropriate dual-space to perform the half-plane query. We map each segment $s$ to a dual-point with coordinates $(b, a + (1-\rho)l)$. Segments matching the majority expression then correspond to the dual-points satisfying the half-plane
\[
y > \frac{1}{\alpha} - \frac{\beta}{\alpha}
\]

%------------------------------------------------------------------------------

\paragraph{Set $S_R$, left endpoint right of $L_1$.}
There are 4 subclasses of segments whose left-endpoints are right of $L_1$, as shown in Figure \ref{fig:slabs:one:aoq_left_l2}.

\begin{figure}[t]
\begin{center}
  \includegraphics[width=0.80\textwidth]{figures/fig_aoq_left_l2}
  \caption{Segments whose left endpoints are between $L_1$ and $L_2$.}
  \label{fig:slabs:one:aoq_left_l2}
\end{center}
\end{figure}


\begin{enumerate}
 \item The entire segment is between $L_1$ and $L_2$. Since $s \in Q \imply s \in_\rho Q$, this segment should be counted.

 \item The entire segment is right of $L_2$. No part of $s$ is in $Q$, so it should not be counted.

 \item The segment intersects $L_2$, and $s \in_\rho Q$.

 \item The segment intersects $L_2$, but $s \not \in_\rho Q$.

\end{enumerate}

For these segments, we test their majority property by checking that ``enough of the segment is inside of $Q$''.  Just as we did with the segments of $S_L$, let $\iline{s}$ be the horizontal line through $s$.  Let $(a'', b)$ be the intersection point of $\iline{s}$ with $L_2$. Solving their respective equations gives $a'' = \frac{b - \beta}{\alpha} + w$.  We will show:
\[
a'' - a \geq \rho l \imply s \in_\rho Q
\]

\begin{proof}
We look at each subcase. If $s \in S_R$ belongs to subcase 1, we have that $a'' - a > l$, which is certainly greater than $\rho l$. If $s$ belongs to subcase 2, then $a'' - a < 0$ since $a$ is right of $a''$, and $s$ will not be counted.  Finally, the correctness of this test is straight-forward when $s$ is in either subcase 3 or 4, since the expression directly measures $|s \cap Q|$.
\end{proof}

We test for segments satisfying $a'' - a \geq \rho l$ using a half-plane query.  Expanding this expression gives:
\[
\begin{split}
a'' - a &\geq \rho l \\
%
\frac{b - \beta}{\alpha} + w - a &\geq \rho l \\
%
\rho l + a &\leq \frac{1}{\alpha} b - \frac{\beta}{\alpha} + w
%
\end{split}
\]

From this inequality, we can construct an appropriate dual-space to perform the half-lane query.  We map each segment $s$ to a dual-point with coordinates $(b, \rho l + a)$. Segments matching the majority expression then correspond to the dual-points satisfying the half-plane 
\[
y \leq \frac{1}{\alpha} x - \frac{\beta}{\alpha} + w
\]

%------------------------------------------------------------------------------
\subsection{Construction and Analysis}
\label{:slabs:one:analysis}

As mentioned in Section \ref{:slabs:one:approach}, we will use the multi-level canonical sets data structure described in Section~\ref{:prelim:chan} to perform parts of this query.  This structure is well-suited to simplicial queries, which is why we decomposed our query region as we did in the previous section. To describe and analyze the structure, it is convenient to look at the innermost structures first, and then use those results to build up the outer structures.


%------------------------------------------------------------------------------
\subsubsection{Length restriction}

The innermost structure answers the length restriction query on a set $S$ of $n$ horizontal segments. This is easily answered using constructing a 1-dimensional range tree, as described in Section~\ref{:prelim:range-trees}, keyed on the segment lengths. The following lemma summarizes the time and space requirements of this structure.

\begin{lemma}
\label{lem:slabs:one:step1}
Given a set of $n$ horizontal line segments, we can count all segments with length at most that of a given query parameter in $\BigOh{\log{n}}$ time, using a data structure of size $\BigOh{n}$, which can be built in $\BigOh{n \log{n}}$ preprocessing time. With the same structure, we can report matching segments in $\BigOh{\log{n} + r}$ time, where $r$ is the number of reported elements.
\end{lemma}


% -----------------------------------------------------------------------------
\subsubsection{Majority Property}

To query segments having the majority property, we use the half-plane expressions we developed in Section \ref{:slabs:one:details:majority}.  There are two expressions, depending on whether the segment belongs in $Q_1$ or $Q_{2a}/Q_{2b}$. For each expression, we construct a separate data structure. We will examine $Q_1$ specifically, but the case for $Q_{2a}/Q_{2b}$ is similar.

During the preprocessing step, each segment $s_i \in S$, $1 \leq i \leq n$ is mapped to a point $p_i$ located at $(x, bx - a - (1 - \rho)l)$. Let $P_1$ be the set of all such points. We perform a half-plane query with the line $y = \alpha\cdot x + \beta$, where $\alpha$ and $\beta$ are query inputs.

By Theorem~\ref{th:chan}, we can answer this query using a canonical subsets data structure for locating points within a query simplex. This structure consists of $k = \BigOh{n}$ canonical subsets, $C_1$ through $C_k$. With each $C_i$, we will associate the structure given by Lemma \ref{lem:slabs:one:step1}, constructed on the elements of that subset.

The space requirements include $\BigOh{n \log{n}}$ for the canonical subsets structure, plus
\[
\sum_{i}^{k}{|C_i|} \leq \BigOh{n\log{n}}
\]

\noindent for the associate data structures, for a combined space requirement of $\BigOh{n\log{n}}$.

Preprocessing time includes $\BigOh{n\log{n}}$ for the canonical subsets structure, plus
\[
\sum_{i}^{k}{|C_i| \log{|C_i|}} 
\leq \sum_{i}^{k}{|C_i| \log{n}} 
\leq \BigOh{n\log{n}} \log{n} 
\leq \BigOh{n \log^2{n}}
\]

\noindent for the associate data structures. The preprocessing time for the associate structures dominates that of the canonical subsets for a combined preprocessing time of $\BigOh{n\log^2{n}}$.

Performing the majority property query against this structure requires $\BigOh{\sqrt{n}\log{n}}$ time and returns $k' = \BigOh{\sqrt{n}\log{n}}$ disjoint canonical sets. For each returned subset, the associated structure for length restriction must be queried, which requires the following query time:
\[
\sum_{i}^{k'}{\log{|C_i|}} 
\leq \sum_{i}^{k'}{\log{n}} 
\leq \log{n} \cdot \sum_{i}^{k'}{1}
\leq \log{n} \BigOh{\sqrt{n}\log{n}}
\leq \BigOh{\sqrt{n}\log^2{n}}
\]

\noindent
for a total query time of $\BigOh{\sqrt{n}\log{n}} + \BigOh{\sqrt{n}\log^2{n}} = \BigOh{\sqrt{n}\log^2{n}}$. The following lemma summarizes our results so far.

\begin{lemma}
\label{lem:slabs:one:step2}
Given a set of $n$ horizontal line segments, we can locate all segments which are not ``too much'' to one side of a query line \emph{and} which are not too long in $\BigOh{\sqrt{n}\log^2{n}}$ time, using a data structure of size $\BigOh{n\log{n}}$, which can be built in $\BigOh{n\log^2{n}}$ time.
\end{lemma}


% -----------------------------------------------------------------------------
\subsubsection{Left-endpoint classification}

To classify the left-endpoints of the segments into regions $Q_1$, $Q_{2a}$, and $Q_{2b}$, we use 3 simplex range searches. By Theorem~\ref{th:chan}, we can locate all left endpoints in $Q_1$ in $\BigOh{\sqrt{n}\log{n}}$ time using a data structure of canonical subsets which requires $\BigOh{n \log{n}}$ size and space to construct. A similar structure can be created for $Q_{2a}/Q_{2b}$.

The canonical subsets structure consists of $k = \BigOh{n}$ subsets, and to complete our overall query, we will construct an associated structure as described in Lemma \ref{lem:slabs:one:step2} with each, based on the elements within. Total space needed is $\BigOh{n \log{n}}$ for the canonical subsets, plus:
\[
\sum_{i}^{k}{|C_i| \log{|C_i|}} \leq \BigOh{n\log^2{n}}
\]

\noindent for the associated structures, for a total space requirement of $\BigOh{n\log^2{n}}$.

Preprocessing time required is $\BigOh{n \log{n}}$ for the canonical subsets, plus
\[
\sum_{i}^{k}{|C_i| \log^2{|C_i|}} 
\leq \sum_{i}^{k}{|C_i| \log^2{n}} 
\leq \BigOh{n\log{n}} \log^2{n}
\leq \BigOh{n \log^3{n}}
\]

\noindent
for the associated structures, for a total preprocessing time of $\BigOh{n \log^3{n}}$.

Query time is $\BigOh{\sqrt{n}\log{n}}$ to find the $k'$ disjoint canonical subsets of total size $\BigOh{\sqrt{n}\log{n}}$ representing the left endpoints of interest, plus
\[
\sum_{i}^{k'}{\BigOh{\sqrt{|C_i|}\log^2{|C_i|}}} 
\leq \sum_{i}^{k'}{\BigOh{\sqrt{|C_i|}}\BigOh{\log^2{n}}} 
\leq \BigOh{\sqrt{n}\log^3{n}}
\]

\noindent
for querying the associated structures, for a total query time of $\BigOh{\sqrt{n} \log^3{n}}$. 

\begin{lemma}
\label{lem:slabs:one:step3}
Given a set of $n$ horizontal line segments, we can locate all segments whose left endpoints are found in within a simplex query region, which are not ``too much'' to one side of a query line, \emph{and} which are not too long in $\BigOh{\sqrt{n}\log^3{n}}$ time, using a data structure of size $\BigOh{n\log{n}^2}$, which can be built in $\BigOh{n\log^3{n}}$ time.
\end{lemma}

% -----------------------------------------------------------------------------
\subsubsection{Combining the steps}

The previous three steps combine to create a data structure which can answer a majority range query for any one of the three simplicies that $Q$ is subdivided into as shown in Figure~\ref{fig:slabs:one:classification}.  To answer any query $Q$, we need to preprocess three such data structures, choosing our expressions appropriately for $Q_1$, $Q_{2a}$, and $Q_{2b}$.  Then, during query time, we need to query all three structures, and combine their results. The following theorem summarizes this process.

\begin{theorem}
\label{thm:slabs:one}
Given a set of $n$ horizontal line segments, we can locate all segments which satisfy the majority property for a given query slab in $\BigOh{\sqrt{n}\log^3{n}}$ time, using a data structure of size $\BigOh{n\log^2{n}}$, requiring $\BigOh{n\log^3{n}}$ preprocessing time.
\end{theorem}


%------------------------------------------------------------------------------
%------------------------------------------------------------------------------
\section{Querying with Two Slabs}
\label{:slabs:two}

We now consider a related instance of the slab query. In this section, the query is input as two slabs, and we are interested in those segments which are sufficiently enclosed by their intersection. When the two slabs are orthogonal to each other, this method will solve the problem of finding segments sufficiently enclosed by an arbitrarily-oriented rectangle.  The overall approach is very similar to Section~\ref{:slabs:one}, and the data structure we will build to answer these queries is again based on the disjoint canonical sets structure outlined in Section~\ref{:prelim:chan}.


% -----------------------------------------------------------------------------
\subsection{Problem Definition}
\label{:slabs:two:problem}

The formal statement of this problem is as follows.

\begin{problem}
We are given a set $S$ of $n$ horizontal line segments in the plane, and a fixed parameter $\rho$ such that $0 < \rho \leq 1$. Our query $Q$ is an arbitrarily-oriented parallelogram specified as the intersection of two slabs. We want to count the number of segments $s \in S$  satisfying the property $|s \cap Q| \geq \rho \cdot |s|$.
\end{problem}

Throughout this section, we use the following definitions. Each segment $s_i \in S, 1 \leq i \leq n$ is defined by three values $a_i, b_i, l_i$ which in turn define the endpoints of the segment as $p_i = (a_i, b_i)$ and $q_i = (a_i + l_i, b_i)$.

A query region is an arbitrarily-oriented parallelogram, which is specified as the intersection of two slabs, $S_p$ and $S_n$, which have positive and negative slopes, respectively. See Figure \ref{fig:slabs:two:ds}.

\begin{figure}[t]
\begin{center}
  \includegraphics[width=0.90\textwidth]{figures/fig_ds_slabs}
  \caption{A query parallelogram $Q$ formed by the inputs $\alpha$, $\beta$, $w_p$,
  $\delta$, $\gamma$, and $w_n$.}
  \label{fig:slabs:two:ds}
\end{center}
\end{figure}

Specifically, a query is given as six inputs $\alpha, \beta, w_p, \delta, \gamma, w_n$, where $\alpha > 0$, $w_p > 0$, $w_n > 0$, and $\delta < 0$. With these inputs, we define:

\begin{itemize}
 \item $L_1 : y = \alpha x + \beta$, the left edge of $S_p$,

 \item $L_2 : y = \alpha (x - w_p) + \beta$, the right edge of $S_p$,

 \item $L_3 : y = \gamma x + \delta$, the left edge of $S_n$,

 \item $L_4 : y = \gamma (x + w_n) + \delta$, the right edge of $S_n$,

 \item $S_p = \{ p \in \mathbb{R}^2 | (p \text{ is on or right of } L_1) \wedge (p \text{ is on or left of } L_2) \}$, a slab with positive slope,

 \item $S_n = \{ p \in \mathbb{R}^2 | (p \text{ is on or right of } L_3) \wedge (p \text{ is on or left of } L_4) \}$, a slab with negative slope,

 \item $Q = S_p \cap S_n$, the query parallelogram.

\end{itemize}

A segment $s \in S$ with length $l$ is sufficiently enclosed in $Q$ if and only if at least $\rho \cdot l$ of the segment is within $Q$. More precisely, we say that $s \in_\rho Q \Leftrightarrow |s \cap Q| \geq \rho \cdot l$. Otherwise, $s \not \in_\rho Q$.


% -----------------------------------------------------------------------------
\subsection{Identifying the Segments}
\label{:slabs:two:approach}

Just as in the single slab problem, identifying which segments meet our enclosure property is accomplished ignoring some segments, classifying the remaining segments, and developing an appropriate majority property expression.

\begin{enumerate}
 \item Decompose the parallelogram into three query subregions based on $y$-coordinates, and isolate those segments which may interact with each subregion.

 \item For each query subregion:
 \begin{enumerate}
  \item Identify the two lines out of $L_1, L_2, L_3,$ and $L_4$ which border the subregion,
  \item Classify segments based on the location of their endpoints with respect to the border lines,
  \item Check the appropriate majority property.
 \end{enumerate}
\end{enumerate}

To answer these queries, we will use a multi-level \emph{canonical sets} as described in Section~\ref{:prelim:chan}. We first describe the different steps of the query in more detail, and then describe the construction and analysis of the query data structure.


% -----------------------------------------------------------------------------
\subsubsection{Decomposition}

\begin{figure}
\begin{center}
  \includegraphics[width=0.85\textwidth]{figures/fig_ds_wpwn}
  \caption[Decomposition of the query region $Q$]{Decomposition of the query region $Q$. The orientation of $Q$ depends on the relative widths of the slabs which define it.}
  \label{fig:slabs:two:wpwn}
\end{center}
\end{figure}

This first step is not found in the single-slab version of the problem. The query region, $Q$, is decomposed into three subregions, $Q_a$, $Q_b$, and $Q_c$, by extending horizontal lines through each vertex of $Q$. These lines determine three horizontal strips $Y_a$, $Y_b$, and $Y_c$, each containing their respective subregion of $Q$. See Figure~\ref{fig:slabs:two:wpwn}.

The widths of the two slabs created by the query inputs determine the overall orientation of $Q$. When $w_p < w_n$, the center parallelogram, $Q_b$, is defined by $L_1$ and $L_2$, and when $w_p > w_n$, $Q_b$ is defined by $L_3$ and $L_4$. When $w_p = w_n$, $Q_b$ disappears, and we don't need to continue further query steps on it.

After partitioning the segments according to whether they fall into $Y_a$, $Y_b$, or $Y_c$, we continue the query for each partition against the appropriate query subregion.


\subsubsection{Querying the Subregions}

Each subregion is determined by two lines based on our query inputs. For $Q_a$, it is always $L_1$ on the left and $L_4$ on the right. Likewise, for $Q_c$, it is always $L_3$ on the left and $L_2$ on the right. For $Q_b$, it is either $L_1$ and $L_2$ or $L_3$ and $L_4$ depending on the relationship of $w_p$ and $w_n$.

We will examine $Q_a$ specifically, but the solutions for the other two zones are quite similar, differing only by the equations of the lines which define the subregion.

The solution for $Q_b$ can either be solved as detailed below, or as an instance of the single slab problem presented in Section~\ref{:slabs:one}.


\subsubsection{Endpoint Classification}

Let $s \in S$ be a horizontal line segment defined by $(a, b, l)$.  Let $\overline{s}$ be the line through $s$, defined by the equation $y = b$.  Let $(a', b) = \overline{s} \cap L_1$ and $(a'', b) = \overline{s} \cap L_4$, be the intersection points of $\overline{s}$ with $L_1$ and $L_4$, respectively, then $a' = \frac{b - \beta}{\alpha}$ and $a'' = \frac{b - \delta}{\gamma} + w_n$.

We classify the endpoints of $s$, namely $p = (a, b)$ and $q = (a + l, b)$ with respect to their position left or right of $L_1$ and $L_4$. We can accomplish this with simplex range searching. Since $b$ is common to all points, we are interested in the following relationships:

\begin{enumerate}
 \item If $a < a'$ and

 \begin{enumerate}
  \item If $a + l < a'$, then $s$ is entirely left of $Q_a$, and is not counted.
  \item If $a' < a + l < a''$, then $s$ intersects $L_1$ only.
  \item If $a'' < a + l$, then $s$ intersects $L_1$ and $L_4$.
 \end{enumerate}

 \item If $a' < a < a''$ and
 \begin{enumerate}
  \item If $a' < a + l < a''$, then $s$ is entirely within $Q_a$, thus $s \in_\rho Q_a$, and it is counted.
  \item If $a'' < a + l$, then $s$ intersects $L_4$ only.
 \end{enumerate}

 \item If $a'' < a$ and
 \begin{enumerate}
  \item If $a'' < a + l$, then $s$ is entirely right of $Q_a$, and is not counted.
 \end{enumerate}
\end{enumerate}

None of the segments in Cases 1(a) and 3(a) are counted, and all of the segments in Case 2(a) \emph{are} counted. All segments falling into cases $1(b)$, $1(c)$, and $2(b)$ need to be tested for their majority property.


\subsubsection{Majority Property}

For Case 1(b), we know that $s$ only crosses $L_1$. In this case, we can either check that enough of $s$ is inside $Q_a$, or we can check that not too much of $s$ is \emph{outside} of $Q_a$.  We will check the latter condition, which gives the following expression.
\[
\begin{split}
a' - a &< (1 - \rho)l \\
%
\frac{b - \beta}{\alpha} - a &< (1 - \rho)l \\
%
b \frac{1}{\alpha} - \frac{\beta}{\alpha} &< a + (1 - \rho)l \\
%
\end{split}
\]

We can query for segments matching this expression by performing a half-plane query on an appropriate dual-space, just as we did in Section~\ref{:slabs:one:details:majority}. For this expression in particular, we map each segment $s$ to a dual-point with coordinates $(b, a + (1-\rho)l)$, and query with the half-plane $y > \frac{1}{\alpha} - \frac{\beta}{\alpha}$.

For Case 2(b), where $s$ crosses only $L_4$ we may check that enough of $s$ is inside $Q_a$, or that not too much is outside of $Q_a$. In this case, we will check the former condition, which gives the following expression.
\[
\begin{split}
a'' - a &\geq \rho l \\
%
\frac{b - \delta}{\gamma} + w_n - a &\geq \rho l \\
%
b \cdot \frac{1}{\gamma} - \frac{\delta}{\gamma} + w_n &\geq a + \rho l \\
%
\end{split}
\]

This expression can be queried by mapping each segment $s$ to a dual-point with coordinates $(b, a + \rho l)$ and locating those which satisfy the half-plane $y \leq \frac{1}{\gamma} x - \frac{\delta}{\gamma} + w_n$.

Finally, for Case 1(c), we know that both endpoints of $s$ are outside of $Q_a$, so we only need to measure the width of $s \cap Q_a$.  Specifically, we require that:
\[
\begin{split} 
a'' - a' &\geq \rho l \\
%
\frac{b - \delta}{\gamma} + w_n - \frac{b - \beta}{\alpha} &\geq \rho l \\
%
\frac{b}{\gamma} - \frac{\delta}{\gamma} + w_n - \frac{b}{\alpha} + \frac{\beta}{\alpha} &\geq \rho l \\
%
b \cdot \left ( \frac{1}{\gamma} - \frac{1}{\alpha} \right ) + \left ( \frac{\beta}{\alpha} - \frac{\delta}{\gamma} + w_n \right ) &\geq \rho l \\
%
\end{split}
\]

This expression can be queried by mapping each segment $s$ to a dual-point with coordinates $(b, \rho l)$ and locating those which satisfy the half-plane
\[
y \leq \left ( \frac{1}{\gamma} - \frac{1}{\alpha} \right ) \cdot x + \left ( \frac{\beta}{\alpha} - \frac{\delta}{\gamma} + w_n \right )
\]

\noindent where the expressions consisting only of query variables can be calculated in $\BigOh{1}$ time during the query.


% -----------------------------------------------------------------------------
\subsection{Construction and Analysis}
\label{:slabs:two:analysis}

We will use the multi-level canonical sets data structure described in Section~\ref{:prelim:chan} for parts of this query, just as we did for the single slab query.  Each of the 4 steps given in Section~\ref{:slabs:two:approach} correspond to one nested level of the data structure.  It is easiest to describe the structure inside-out, so we begin with the innermost structure.


\subsubsection{Check Majority Property}

To check for segments having the majority property, we use the half-plane dual-spaces we developed in Section~\ref{:slabs:two:approach}. By Theorem~\ref{:prelim:chan}, we can answer such a half-plane query directly using a canonical subsets data structure, yielding the following lemma.

\begin{lemma}
\label{lem:slabs:two:step1}
Given a set of $n$ horizontal line segments, we can locate all segments which are not ``too much'' to one side of a query line in $\BigOh{\sqrt{n}\log{n}}$ time, using a data structure of size $\BigOh{n\log{n}}$, which can be built in $\BigOh{n\log{n}}$ time.
\end{lemma}


\subsubsection{Classify Right Endpoint}

We need to be able to classify the right endpoints of our segments into one of several simplicies. These cases are:
\begin{itemize}
 \item Within the triangle $Q_a$.
 \item Within the parallelogram $Q_b$, which can be decomposed into two triangles at query time.
 \item Within the triangle $Q_c$.
 \item To the right of $L_4$.
 \item To the right of $L_2$.
\end{itemize}

Each of these simplex queries can be answered with a disjoint canonical subsets data structure. Such a structure will consist of $k = \BigOh{n}$ canonical subsets, $C_1$ through $C_k$. With each $C_i$, we will associate the structure from Lemma~\ref{lem:slabs:two:step1}, constructed on the elements of that subset.

The space required includes $\BigOh{n \log{n}}$ for the canonical subsets structure, plus
\[
\sum_{i}^{k}{|C_i| \log{|C_i|}} 
\leq \sum_{i}^{k}{|C_i| \log{n}} 
\leq \BigOh{n\log{n}} \log{n} 
\leq \BigOh{n \log^2{n}}
\]

\noindent for the associate data structures, for a combined space requirement of $\BigOh{n\log^2{n}}$. Preprocessing time is calculated using identical reasoning and is also $\BigOh{n \log^2{n}}$ time.

Performing a right endpoint classification query requires $\BigOh{\sqrt{n}\log{n}}$ time and returns $k' = \BigOh{\sqrt{n}\log{n}}$ disjoint canonical sets. For each returned subset, the associated structure from Lemma~\ref{lem:slabs:two:step1} must be queried, which requires the following additional query time:
\[
\begin{split}
\sum_{i}^{k'}{\BigOh{\sqrt{|C_i|}\log{|C_i|}}} 
&\leq \sum_{i}^{k'}{\BigOh{\sqrt{|C_i|} \log{n}}} \\
%
&\leq \log{n} \cdot \sum_{i}^{k'}{\BigOh{\sqrt{|C_i|}}} \\
%
&\leq \log{n} \cdot \BigOh{\sqrt{n}\log{n}} \\
%
&\leq \BigOh{\sqrt{n}\log^2{n}} \\
%
\end{split}
\]

\noindent for a total query time of $O(\sqrt{n}\log{n}) + O(\sqrt{n}\log^2{n}) = O(\sqrt{n}\log^2{n})$. The following lemma summarizes our results so far.

\begin{lemma}
\label{lem:slabs:two:step2}
Given a set of $n$ horizontal line segments, we can locate all segments which have their right endpoint in a query simplex and which are not ``too much'' to one side of a query line in $\BigOh{\sqrt{n}\log^2{n}}$ time, using a data structure of size $\BigOh{n\log^2{n}}$, which can be built in $\BigOh{n\log^2{n}}$ time.
\end{lemma}


\subsubsection{Classify Left Endpoint}

We need to be able to classify the left endpoints of our segments as well. We are interested in endpoints appearing in the following locations:

\begin{itemize}
 \item To the left of $L_1$.
 \item To the left of $L_3$.
 \item Within the triangle $Q_a$.
 \item Within the parallelogram $Q_b$, which can be decomposed into two triangles at query time.
 \item Within the triangle $Q_c$.
\end{itemize}

This level of the nested disjoint canonical subsets structure will be built just as in the last two steps.  We first construct the structure with respect to the left endpoints.  Such a structure will consist of $k = \BigOh{n}$ canonical subsets, $C_1$ through $C_k$. With each $C_i$, we will associate the structure from Lemma~\ref{lem:slabs:two:step2}, constructed on the elements of that subset.

Through similar reasoning as in the the previous step, the preprocessing time and space requirements, as well as the query time, will each gain an additional factor $\BigOh{\log{n}}$. The following lemma summarizes the data structure that we produce at this step.

\begin{lemma}
\label{lem:slabs:two:step3}
Given a set of $n$ horizontal line segments, we can locate all segments which have their left and right endpoints in given query simplicies and which are not ``too much'' to one side of a query line in $\BigOh{\sqrt{n}\log^3{n}}$ time, using a data structure of size $\BigOh{n\log^3{n}}$, which can be built in $\BigOh{n\log^3{n}}$ time.
\end{lemma}


\subsubsection{Decomposition}

The outermost level of the structure classifies segments according to their $y$-coordinate. This is a straight-forward application of an orthogonal range query tree as outlined in Section~\ref{:prelim:range-trees}.  Any range tree on the $y$-coordinates of segments can be queried for segments falling into either $Y_a, Y_b, Y_c$. The following lemma summarizes the time and space requirements of this step.

XXX TODO what is the analysis here?  Update the lemma and the following theorem

\begin{lemma}
\label{lem:slabs:two:step4}
Given a set of $n$ horizontal line segments, we can count all segments whose $y$-coordinate falls between two values, whose left and right endpoints in given query simplicies, and which are not ``too much'' to one side of a query line in $\BigOh{\sqrt{n}\log^4{n}}$ time, using a data structure of size $\BigOh{n\log^4{n}}$, which can be built in $\BigOh{n\log^4{n}}$ time.
\end{lemma}


\subsubsection{Combining the steps}

The previous steps combine to create a data structure which can answer a majority range query for parallelogram formed by the intersection of two slabs. To answer a query $Q$, we need to preprocess several versions of this datastructure, one for each case.

 three such data structures, choosing our expressions appropriately for $Q_1$, $Q_{2a}$, and $Q_{2b}$.  Then, during query time, we need to query all three structures, and combine their results. The following theorem summarizes this process.

\begin{theorem}
\label{thm:slabs:two}
Given a set of $n$ horizontal line segments, we can locate all segments which satisfy the majority property with respect to the intersection of two query slabs in $\BigOh{\sqrt{n}\log^4{n}}$ time, using a data structure of size $\BigOh{n\log^3{n}}$, requiring $\BigOh{n\log^4{n}}$ preprocessing time.
\end{theorem}



%------------------------------------------------------------------------------
%------------------------------------------------------------------------------
\section{Conclusion}
\label{:slabs:concl}

XXX TODO

Open Problems

* Get rid of the 1/2 limitation on the two slab case.

* Arbitrarily-oriented segments. Not conceptually hard, but quickly turn into higher dimensional problems

